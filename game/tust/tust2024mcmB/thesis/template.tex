%! Mode:: "TeX:UTF-8"
%! TEX program = xelatex
\PassOptionsToPackage{quiet}{xeCJK}
\documentclass[bwprint]{cumcmthesis}
\usepackage{etoolbox}
\BeforeBeginEnvironment{tabular}{\zihao{-5}}
\usepackage[numbers,sort&compress]{natbib}  % 文献管理宏包
\usepackage[framemethod=TikZ]{mdframed}  % 框架宏包
\usepackage{url}  % 网页链接宏包
\usepackage{subcaption}  % 子图宏包
\newcolumntype{C}{>{\centering\arraybackslash}X}
\newcolumntype{R}{>{\raggedleft\arraybackslash}X}
\newcolumntype{L}{>{\raggedright\arraybackslash}X}

% 添加代码高亮支持
\usepackage{listings}
\usepackage{xcolor}

% 定义代码美化样式
\definecolor{codebackground}{rgb}{0.95,0.95,0.95}
\definecolor{codekeyword}{rgb}{0.0,0.0,0.6}
\definecolor{codecomment}{rgb}{0.0,0.5,0.0}
\definecolor{codestring}{rgb}{0.58,0.0,0.82}

\lstdefinestyle{pythonstyle}{
    backgroundcolor=\color{codebackground},   
   % basicstyle=\ttfamily\small,
    basicstyle=\small\fontspec{DejaVu Sans Mono}, % 或者 Consolas, Noto Mono, Liberation Mono, Source Code Pro
    breakatwhitespace=false,
    breaklines=true,
    captionpos=t,
    commentstyle=\color{codecomment},
    keepspaces=true,
    keywordstyle=\color{codekeyword}\bfseries,
    language=Python,
    numberstyle=\tiny\color{black},
    rulecolor=\color{black},
    showspaces=false,
    showstringspaces=false,
    showtabs=false,
    stringstyle=\color{codestring},
    tabsize=4,
    frame=single,
    framesep=8pt,
    framexleftmargin=10pt,
    framexrightmargin=10pt,
    xleftmargin=10pt,
    xrightmargin=10pt,
    numbers=left,
    numbersep=5pt,
}

\lstset{style=pythonstyle}

% 自定义信息
\newcommand{\studentid}{24103219} % 学号
\newcommand{\studentname}{贺江阳} % 姓名
\newcommand{\college}{人工智能学院} % 所在学院
\newcommand{\major}{软件工程} % 专业

% 重定义章节标题格式
\renewcommand{\thesection}{\chinese{section}、}
\renewcommand{\thesubsection}{\arabic{section}.\arabic{subsection}}
\renewcommand{\thesubsubsection}{\arabic{section}.\arabic{subsection}.\arabic{subsubsection}}

% 自定义首页格式
\renewcommand{\maketitle}{
  \begin{titlepage}
    \begin{center}
      \vspace*{2cm}
      {\huge\bfseries 基于层次分析和模糊综合评价的翻译软件双向翻译质量评价模型研究\par}
      \vspace{2cm}
      
      \begin{tabular}{rl}
        \Large\textbf{学 \quad 号:} & \Large\underline{\makebox[8cm]{\studentid}} \\[12pt]
        \Large\textbf{姓 \quad 名:} & \Large\underline{\makebox[8cm]{\studentname}} \\[12pt]
        \Large\textbf{所在学院:} & \Large\underline{\makebox[8cm]{\college}} \\[12pt]
        \Large\textbf{专 \quad 业:} & \Large\underline{\makebox[8cm]{\major}} \\[12pt]
      \end{tabular}
      
    \end{center}
  \end{titlepage}
}

% 重新定义目录样式
\renewcommand{\contentsname}{目录}

%%%%%%%%%%%%%%%%%%%%%%%%%%%%%%%%%%%%%%%%%%%%%%%%%%%%%%%%%%%%%
%% 正文
\begin{document}

% 显示封面
\maketitle

% 添加目录页
\newpage
\tableofcontents  % 目录
\thispagestyle{plain}
\clearpage
\newpage
\pagenumbering{arabic}

%%%%%%%%%%%%%%%%%%%%%%%%%%%%%%%%%%%%%%%%%%%%%%%%%%%%%%%%%%%%% 
% 摘要页
\section*{摘要}
\addcontentsline{toc}{section}{摘要}

本文针对翻译软件评价问题,构建了一个基于层次分析法(AHP)和模糊综合评价的数学模型,通过"双向翻译"的方式对三款主流翻译软件进行了全面评估。模型设计了包括语义保留度(SP)、语法正确性(GC)、表达流畅性(EF)和专业术语准确性(TA)在内的多维评价指标体系,并通过余弦相似度、编辑距离等算法实现了客观量化评估。

针对问题一,本文对三款翻译软件(Google翻译、百度翻译和DeepL翻译)进行了对比评测。评价结果表明,Google翻译在总体表现上最为优异,总分达到0.7574,在语义保留度(0.8695)和语法正确性(1.0000)两个维度表现突出。通过模糊综合评价方法,进一步验证了Google翻译在"优"等级上具有显著优势,加权得分为0.7091。

针对问题二,本文选取了一篇关于翻译重要性的英文短文进行双向翻译测试。结果显示三款软件在英文文本翻译中表现相近,其中百度翻译略胜一筹(总分0.6947),但差异不显著。这表明当前翻译软件在处理规范英文文本时已经达到了相当的水平。

针对问题三,本文选择表现最好的Google翻译,对四种不同类型的中文文本进行了评测。结果表明翻译软件对不同类型文本的处理能力存在显著差异:专业课程介绍和数学教材的翻译效果最好(分别为0.8035和0.7706),而对古散文(0.5643)和唐诗(0.4436)的翻译效果相对较差。这反映出现代翻译软件在处理文学性强、语言精炼的古典文本时仍面临挑战。

本文的研究表明,虽然当前翻译软件在处理现代通用文本时已经达到较好水平,但在处理特定类型文本时仍存在明显局限。基于研究结果,本文对翻译软件的发展方向提出了建议,为提高机器翻译质量提供了参考依据。

\keywords{层次分析法\quad 模糊综合评价\quad 双向翻译\quad 机器翻译\quad 文本相似度}




%%%%%%%%%%%%%%%%%%%%%%%%%%%%%%%%%%%%%%%%%%%%%%%%%%%%%%%%%%%%%  

\section{问题的提出}
\subsection{问题的重述}

作为国家选拔人才的重要方式,高考关系着国家未来的发展,更是关系着每一个考生和几乎每一个家庭的切身利益。而对于如何选拔人才,选拔什么样的人才,高考的内容尤其是科目一直被广泛地热议,其中做为高考主科的英语(外语),一直处于讨论的"风口浪尖"。很多人建议高考减少英语的占比,甚至建议取消高考英语,其中的一个重要的理由是"随着翻译软件的不断成熟和发展,翻译工作可以教给计算机完成,所以不需要'全民'学英语",但不可否认的是,现在的翻译软件还不能完全代替人工。

对于翻译软件的评价有一个常用的方法就是"双向翻译",即选取一段文字完成"中译英"后,将翻译的结果再"英译中",再与原文进行比对。本文将针对这种比对方式,建立数学模型,选取三款翻译软件(在线翻译或移动端APP或应用程序)对翻译效果进行评价,并完成以下问题:

\textbf{问题一:} 建立数学模型,针对附件中的一段文字,完成"双向翻译"后,对着三款翻译软件进行对比,评价。

\textbf{问题二:} 利用问题一所建立的数学模型,选取一篇学过的英文短文进行双向翻译,并对翻译效果进行评价。

\textbf{问题三:} 综合问题一和问题二选取效果最好的一款软件,利用问题一所建立的数学模型,选取(1)一首唐诗或宋词;(2)一段古散文;(3)高等数学教材的一段叙述性文字;(4)所学的某一门专业课的课程简介,进行"双向翻译",对上述四种类型的文字翻译效果进行评价。

\textbf{问题四:} 针对上述的建模结果,分析在将来翻译软件是否可以替代人工,以及目前的翻译软件还有哪些方面可以进行改进。

\subsection{研究背景}

\textbf{(1) 研究意义}

随着全球化进程的加速,语言翻译在跨文化交流、国际贸易、学术交流等领域变得日益重要。近年来,机器翻译技术取得了显著进步,从早期的基于规则和统计的方法发展到如今的神经机器翻译(NMT),翻译软件的功能和质量有了极大提升。这引发了对翻译软件是否能够部分或完全取代人工翻译的广泛讨论,尤其是在教育领域,引发了对外语学习必要性的重新思考。

对翻译软件质量进行科学、客观的评价,不仅能够帮助用户选择适合自己需求的工具,也能够为翻译软件的开发者提供改进方向,促进机器翻译技术的发展。而通过双向翻译的评价方法,能够更全面地检验翻译软件的准确性和可靠性,为评估翻译软件在不同应用场景下的实用价值提供依据。

\textbf{(2) 参考文献综述}

机器翻译的评价一直是自然语言处理领域的重要研究课题。Papineni等人\cite{papineni2002bleu}提出了BLEU(Bilingual Evaluation Understudy)评分系统,将机器翻译的输出与人工翻译的参考译文进行比较,成为当前最广泛使用的机器翻译自动评价指标之一。此后,Lin和Och\cite{lin2004orange}提出了ROUGE(Recall-Oriented Understudy for Gisting Evaluation)评分系统,更加侧重于召回率的评估。

近年来,随着深度学习在自然语言处理领域的应用,机器翻译的质量评价方法也得到了进一步发展。Zhang等人\cite{zhang2020bertscore}提出了BERTScore,利用预训练语言模型BERT提取上下文特征来比较机器翻译与参考译文的语义相似度,解决了传统基于n-gram匹配方法难以捕捉语义信息的问题。而Rei等人\cite{rei2020comet}提出的COMET评价模型,则通过神经网络学习人类评价者的判断标准,进一步提高了评价的准确性。

在评价框架方面,傅悦\cite{fu2019translation}基于模糊综合评价方法构建了多层次、多指标的翻译质量评价体系,综合考虑了忠实度、表达度和规范度等因素。而刘群和李向阳\cite{liu2010comprehensive}则提出了结合层次分析法(AHP)和主成分分析的综合评价方法,为翻译软件的客观评价提供了新的思路。

本文借鉴上述研究成果,基于双向翻译的特点,构建了一个结合层次分析法和模糊综合评价的数学模型,对翻译软件的质量进行全面、客观的评价。

%%%%%%%%%%%%%%%%%%%%%%%%%%%%%%%%%%%%%%%%%%%%%%%%%%%%%%%%%%%%% 

\section{问题的分析}
对本文提出的关于翻译软件评价的问题,我逐一做如下分析:

\textbf{问题一的分析:}问题一要求建立数学模型评价三款翻译软件的"双向翻译"效果。关键在于如何构建一个科学、客观的评价指标体系,并对各指标进行合理量化。为此,需要从翻译质量的多个方面进行考量,如语义保留度、语法正确性、表达流畅性和专业术语准确性等。同时,不同指标对翻译质量的影响程度不同,需要采用层次分析法确定各指标的权重。考虑到评价过程中的模糊性和不确定性,还需引入模糊综合评价方法,以提高评价结果的客观性和可靠性。具体求解过程中,将通过文本相似度计算、语法规则检查等方法对各指标进行量化,并通过加权综合得出总体评价结果。

\textbf{问题二的分析:}问题二要求将问题一建立的模型应用于英文短文的双向翻译评价。这一问题的特点在于文本类型的变化,从中文文本转为英文文本。这一变化可能导致评价指标的适用性产生变化,因此需要检验模型对不同语种文本的适应性,并可能对某些指标计算方法进行适当调整。例如,对于英文文本,语法规则和表达习惯与中文存在较大差异,可能需要采用不同的语法正确性和表达流畅性评价标准。通过这一问题的解决,可以验证模型的泛化能力和稳定性。

\textbf{问题三的分析:}问题三要求对不同类型的中文文本(唐诗/宋词、古散文、数学教材、专业课程)进行评价。这些文本类型在语言风格、专业术语、表达方式等方面存在显著差异,对翻译软件的性能提出了更高要求。通过评价不同类型文本的翻译效果,可以全面考察翻译软件的能力边界和适用范围。在解决这一问题时,需要针对不同类型文本的特点调整专业术语词典,并关注翻译软件在处理古典文学、专业学术文本等具有挑战性的文本时的表现。这将为问题四的分析提供重要依据。

%%%%%%%%%%%%%%%%%%%%%%%%%%%%%%%%%%%%%%%%%%%%%%%%%%%%%%%%%%%%% 

\section{模型假设}

根据本文提出的问题和以上的问题分析,我做了如下模型假设:

\begin{itemize}[itemindent=2em]
\item 假设1:原文本和双向翻译后的文本之间的相似度可以客观反映翻译质量的好坏,相似度越高,表明翻译质量越好。
\item 假设2:翻译质量可以通过语义保留度、语法正确性、表达流畅性和专业术语准确性四个维度进行全面评价,这四个维度能够涵盖翻译质量的主要方面。
\item 假设3:各评价指标之间存在一定的相对重要性差异,可以通过层次分析法确定各指标的权重,并且这些权重在评价不同类型文本时保持相对稳定。
\item 假设4:采用的文本相似度算法(如余弦相似度、编辑距离)能够有效地反映两段文本在语义上的相似程度。
\item 假设5:语法正确性和表达流畅性可以通过分析文本的句法结构、词语使用频率、句子长度变化等特征进行客观评价。
\item 假设6:在实验过程中,三款翻译软件的性能保持稳定,不会因外部因素(如网络延迟、版本更新等)而产生显著波动。
\end{itemize}

%%%%%%%%%%%%%%%%%%%%%%%%%%%%%%%%%%%%%%%%%%%%%%%%%%%%%%%%%%%%% 

\section{符号说明}
本文常用符号见下表, 其它符号见文中说明.
\begin{table}[H]
\centering
\begin{tabularx}{\textwidth}{CLC}
\toprule
符号    & 说明    & 单位 \\
\midrule
$S_o$   & 原始文本 & - \\
$S_t$   & 翻译后文本 & - \\
$SP$    & 语义保留度 & - \\
$GC$    & 语法正确性 & - \\
$EF$    & 表达流畅性 & - \\
$TA$    & 专业术语准确性 & - \\
$w_i$   & 指标权重 & - \\
$sim_{cos}(S_1,S_2)$ & 文本$S_1$和$S_2$的余弦相似度 & - \\
$d_{edit}(S_1,S_2)$ & 文本$S_1$和$S_2$的编辑距离 & - \\
$R$ & 模糊关系矩阵 & - \\
$A$ & 判断矩阵 & - \\
$T$ & 总评分 & - \\
\bottomrule
\end{tabularx}
\label{tab:符号说明}
\end{table}

%%%%%%%%%%%%%%%%%%%%%%%%%%%%%%%%%%%%%%%%%%%%%%%%%%%%%%%%%%%%% 

\section{建模与求解}
这里将通过数学建模,讨论/解决本文所提出的问题。共分为三个小节。

\subsection{问题一的建模与求解}
问题一主要要解决如何构建评价翻译软件"双向翻译"效果的数学模型,并使用该模型对三款翻译软件进行评价。为此,我们建立了一个基于层次分析法(AHP)和模糊综合评价的评价模型,综合考虑语义保留度、语法正确性、表达流畅性和专业术语准确性四个关键指标。本小节主要内容包括评价指标体系的建立、层次分析法确定权重、基于多种算法的指标量化方法以及模糊综合评价的实现。

\subsubsection{数据的预处理}
\textbf{1. 数据的采集}
    
为评价翻译软件的"双向翻译"效果,我们选取了三款主流翻译软件:谷歌翻译(Google Translate)、百度翻译和DeepL翻译。这三款翻译软件在市场占有率、技术路线和用户评价等方面各有特点,具有较强的代表性。

我们选取题目中提供的一段关于高考英语讨论的文字作为原始文本,分别使用这三款翻译软件进行"中译英",然后再将英文翻译结果进行"英译中",得到双向翻译后的中文文本。采集的数据包括原始中文文本和三款翻译软件双向翻译后的中文文本。
    
\textbf{2. 文本预处理}
    
为了客观评价翻译效果,我们对文本进行了以下预处理:
\begin{itemize}
    \item 分词处理:使用jieba分词工具对中文文本进行分词,便于后续的文本相似度计算。
    \item 去除停用词:移除常见的停用词(如"的"、"了"、"和"等),减少这些高频但意义较弱的词对评价结果的干扰。
    \item 标点符号处理:统一标点符号格式,减少标点符号差异对评价结果的影响。
    \item 专业术语提取:针对文本内容,建立专业术语词典,用于专业术语准确性评价。
\end{itemize}

预处理后的文本更适合进行客观的相似度计算和评价指标量化。
    
\textbf{3. 专业术语词典构建}
    
针对问题一中的文本内容,我们构建了以下专业术语词典,为每个术语赋予重要性权重:
\begin{table}[H]
\centering
\caption{问题一专业术语词典}
\begin{tabularx}{0.8\textwidth}{XXX}
\toprule
术语 & 重要性权重 & 说明 \\
\midrule
科技大学 & 1.5 & 特定机构名称 \\
高校 & 1.5 & 教育领域专业术语 \\
学科 & 1.5 & 教育领域专业术语 \\
工程 & 1.3 & 专业领域名称 \\
科学 & 1.3 & 专业领域名称 \\
教育 & 1.3 & 专业领域名称 \\
多科性 & 1.4 & 特定描述术语 \\
协调发展 & 1.4 & 特定描述术语 \\
重点建设 & 1.4 & 特定描述术语 \\
\bottomrule
\end{tabularx}
\label{tab:术语词典1}
\end{table}

\subsubsection{翻译质量评价模型的建立}
基于翻译质量评价的理论和实践,我们从四个关键维度构建了评价指标体系,并设计了相应的量化算法。

\textbf{1. 评价指标体系}

我们将翻译质量评价分为四个主要指标:
\begin{itemize}
    \item 语义保留度(SP):评估原文本与双向翻译后文本的语义相似程度。
    \item 语法正确性(GC):评估翻译文本的语法是否正确、合规。
    \item 表达流畅性(EF):评估翻译文本的表达是否自然、流畅。
    \item 专业术语准确性(TA):评估专业术语的翻译是否准确、一致。
\end{itemize}

\textbf{2. 层次分析法确定权重}

为科学确定各指标的权重,我们采用层次分析法(AHP):

首先,构建判断矩阵$A$:
\begin{equation}
A = \begin{pmatrix}
1 & 2 & 2 & 3 \\
1/2 & 1 & 1 & 2 \\
1/2 & 1 & 1 & 1 \\
1/3 & 1/2 & 1 & 1
\end{pmatrix}
\end{equation}

其中,矩阵元素$a_{ij}$表示指标$i$相对于指标$j$的重要程度。

然后,通过求解特征值方程$AW = \lambda_{max}W$,得到权重向量$W$:
\begin{equation}
W = [0.4, 0.25, 0.2, 0.15]^T
\end{equation}

即语义保留度(SP)权重为0.4,语法正确性(GC)权重为0.25,表达流畅性(EF)权重为0.2,专业术语准确性(TA)权重为0.15。

一致性检验结果$CR = 0.015 < 0.1$,表明判断矩阵的一致性可接受,权重分配合理。

\textbf{3. 指标量化模型}

\begin{table}[H]
\centering
\caption{翻译质量评价模型算法流程}
\begin{tabularx}{\textwidth}{X}
\toprule
\textbf{翻译质量评价模型算法流程} \\
\midrule
\textbf{输入:} 原始文本$S_o$,双向翻译后的文本$S_t$,专业术语词典$D$ \\
\textbf{输出:} 综合评价得分$Total$,模糊评价结果$B$ \\
\textbf{步骤1:} 文本预处理 \\
\quad 对$S_o$和$S_t$进行分词处理 \\
\quad 去除停用词 \\
\quad 标点符号统一处理 \\
\textbf{步骤2:} 评价指标计算 \\
\quad 计算语义保留度$SP$ \\
\quad \quad 计算余弦相似度$sim_{cos}(S_o, S_t)$ \\
\quad \quad 计算编辑距离相似度$sim_{edit}(S_o, S_t)$ \\
\quad \quad 合并:$SP = 0.6 \times sim_{cos}(S_o, S_t) + 0.4 \times sim_{edit}(S_o, S_t)$ \\
\quad 计算语法正确性$GC$ \\
\quad 计算表达流畅性$EF$ \\
\quad 计算专业术语准确性$TA$ \\
\textbf{步骤3:} 综合评分计算 \\
\quad $Total = w_{SP} \times SP + w_{GC} \times GC + w_{EF} \times EF + w_{TA} \times TA$ \\
\textbf{步骤4:} 模糊综合评价 \\
\quad 构建模糊关系矩阵$R$ \\
\quad 计算模糊综合评价结果$B = W \circ R$ \\
\quad 计算加权得分$S_{weighted} = \sum_{j=1}^4 B_j \times V_j$ \\
\bottomrule
\end{tabularx}
\label{tab:评价模型算法}
\end{table}

\textbf{(1) 语义保留度(SP)模型}

语义保留度通过计算原文本与双向翻译后文本的相似度来评估。我们结合余弦相似度和编辑距离两种方法:

余弦相似度计算公式:
\begin{equation}
sim_{cos}(S_o, S_t) = \frac{\sum_{i=1}^{n} v_{o,i} \times v_{t,i}}{\sqrt{\sum_{i=1}^{n} v_{o,i}^2} \times \sqrt{\sum_{i=1}^{n} v_{t,i}^2}}
\end{equation}

其中,$v_{o,i}$和$v_{t,i}$分别表示原文本和翻译文本中词语$i$的权重(基于词频-逆文档频率TF-IDF计算)。

编辑距离相似度计算:
\begin{equation}
sim_{edit}(S_o, S_t) = 1 - \frac{d_{edit}(S_o, S_t)}{max(len(S_o), len(S_t))}
\end{equation}

其中,$d_{edit}(S_o, S_t)$表示原文本和翻译文本的Levenshtein距离。

最终语义保留度为两种相似度的加权平均:
\begin{equation}
SP = 0.6 \times sim_{cos}(S_o, S_t) + 0.4 \times sim_{edit}(S_o, S_t)
\end{equation}

\textbf{(2) 语法正确性(GC)模型}

语法正确性通过分析文本的句法结构和标点符号使用来评估:
\begin{equation}
GC = 1.0 - \frac{E_{punct} + E_{struct}}{2 \times max(1, N_{sent})}
\end{equation}

其中,$E_{punct}$表示标点符号错误数量,$E_{struct}$表示句子结构错误数量,$N_{sent}$表示文本的句子数量。

\textbf{(3) 表达流畅性(EF)模型}

表达流畅性通过多个因素综合评估:
\begin{equation}
EF = 0.3L_s + 0.3R_s + 0.15P_d + 0.15C_s + 0.1L_d
\end{equation}

其中,$L_s$表示句子长度得分,$R_s$表示词语重复率得分,$P_d$表示句式多样性得分,$C_s$表示连接词使用得分,$L_d$表示句子长度变化得分。

\textbf{(4) 专业术语准确性(TA)模型}

专业术语准确性通过比较原文和翻译文本中专业术语的保留情况评估:
\begin{equation}
TA = \frac{\sum_{t \in T_p} w_t}{\sum_{t \in T_o} w_t}
\end{equation}

其中,$T_o$表示原文中的专业术语集合,$T_p$表示翻译文本中保留的专业术语集合,$w_t$表示术语$t$的重要性权重。

\textbf{4. 综合评价模型}

综合评分通过加权求和计算:
\begin{equation}
Total = w_{SP} \times SP + w_{GC} \times GC + w_{EF} \times EF + w_{TA} \times TA
\end{equation}

其中,$w_{SP}$、$w_{GC}$、$w_{EF}$、$w_{TA}$分别为四个指标的权重。

\textbf{5. 模糊综合评价模型}

考虑到评价过程的模糊性,我们引入模糊综合评价方法:

首先,建立评价等级集 $V = \{\text{优}, \text{良}, \text{中}, \text{差}\}$。 

然后,构建模糊关系矩阵$R$,矩阵元素$r_{ij}$表示第$i$个指标对第$j$个评价等级的隶属度。

隶属度函数设计如下:
\begin{eqnarray} % 保持 eqnarray 环境
\mu_{\text{优}}(x) &=& % 使用 \text{优}
\begin{cases}
\min(1.0, (x-0.9) \times 10 + 0.9), & x \geq 0.9 \\
(x-0.8) \times 7, & 0.8 \leq x < 0.9 \\
0, & x < 0.8
\end{cases} \\ % eqnarray 需要换行符 \\
\mu_{\text{良}}(x) &=& % 使用 \text{良}
\begin{cases}
\min(1.0, (0.9-x) \times 10 + 0.3), & 0.8 \leq x < 0.9 \\
\min(1.0, (x-0.7) \times 7 + 0.3), & 0.7 \leq x < 0.8 \\
(x-0.6) \times 7, & 0.6 \leq x < 0.7 \\
0, & x < 0.6 \text{ 或 } x \geq 0.9
\end{cases} \\ % eqnarray 需要换行符 \\
\end{eqnarray}

$\mu_{\text{中}}$ 和 $\mu_{\text{差}}$ 的定义类似。 % 行内数学模式同样用 \text{}
\\
最后,通过模糊合成运算$B = W \circ R$得到综合评价结果,其中$W$为权重向量,$\circ$表示模糊合成运算。

加权得分计算:
\begin{equation}
S_{weighted} = \sum_{j=1}^4 B_j \times V_j
\end{equation}

其中,$V = [1.0, 0.75, 0.5, 0.25]$为各等级的分值。

\subsubsection{模型的求解和分析}
\textbf{1. 计算方法与结果}

我们使用Python编程语言实现了上述评价模型,对三款翻译软件的双向翻译结果进行了评价。关键计算步骤包括:
\begin{itemize}
    \item 使用jieba分词工具对文本进行分词和预处理
    \item 计算余弦相似度和编辑距离
    \item 分析文本的语法结构和流畅性
    \item 检测专业术语的翻译准确性
    \item 计算综合评分和模糊综合评价结果
\end{itemize}

评价结果如下表:

\begin{table}[H]
\centering
\caption{问题一:三款翻译软件评价结果}
\begin{tabularx}{\textwidth}{CCCCCCC}
\toprule
翻译软件 & 语义保留度(SP) & 语法正确性(GC) & 表达流畅性(EF) & 术语准确性(TA) & 总分 & 模糊评价加权得分 \\
\midrule
Google & 0.8695 & 1.0000 & 0.2653 & 0.7100 & 0.7574 & 0.7091 \\
Baidu & 0.6322 & 0.6667 & 0.3767 & 0.7100 & 0.6014 & 0.5273 \\
DeepL & 0.8560 & 1.0000 & 0.2666 & 0.5800 & 0.7327 & 0.6832 \\
\bottomrule
\end{tabularx}
\label{tab:问题一评价结果}
\end{table}

翻译软件按总分排名如下:
\begin{enumerate}
    \item Google翻译:0.7574
    \item DeepL翻译:0.7327
    \item 百度翻译:0.6014
\end{enumerate}

详细的模糊综合评价结果如下表:

\begin{table}[H]
\centering
\caption{问题一:三款翻译软件模糊综合评价结果}
\begin{tabularx}{\textwidth}{CCCCC}
\toprule
翻译软件 & 优 & 良 & 中 & 差 \\
\midrule
Google & 0.4283 & 0.2654 & 0.1063 & 0.2000 \\
Baidu & 0.0000 & 0.3093 & 0.4907 & 0.2000 \\
DeepL & 0.3886 & 0.2614 & 0.1217 & 0.2283 \\
\bottomrule
\end{tabularx}
\label{tab:问题一模糊评价}
\end{table}

\begin{table}[H]
\centering
\caption{问题一:三款翻译软件评价结果可视化}
\begin{tabularx}{\textwidth}{CCCCCCC}
\toprule
翻译软件 & 语义保留度 & 语法正确性 & 表达流畅性 & 术语准确性 & 总分 & 模糊评价得分 \\
\midrule
Google & 0.87 & 1.00 & 0.27 & 0.71 & 0.76 & 0.71 \\
DeepL & 0.86 & 1.00 & 0.27 & 0.58 & 0.73 & 0.68 \\
百度 & 0.63 & 0.67 & 0.38 & 0.71 & 0.60 & 0.53 \\
\bottomrule
\end{tabularx}
\label{tab:问题一评价可视化}
\end{table}

\textbf{2. 结果分析}

从评价结果可以看出:

\begin{itemize}
    \item Google翻译在总体表现上最优,总分为0.7574,在语义保留度(0.8695)和语法正确性(1.0000)两个指标上表现尤为突出。模糊综合评价中,Google在"优"等级的隶属度最高(0.4283),加权得分也最高(0.7091)。
    
    \item DeepL翻译紧随其后,总分为0.7327,其语义保留度(0.8560)和语法正确性(1.0000)也很高,但在专业术语准确性(0.5800)方面略有不足。
    
    \item 百度翻译的总体表现相对较弱,总分为0.6014,在语义保留度(0.6322)和语法正确性(0.6667)方面有明显差距,但在表达流畅性(0.3767)方面略优于其他两款软件。
    
    \item 三款翻译软件在表达流畅性方面普遍得分较低,这反映了当前机器翻译在生成自然流畅的表达方面仍有较大提升空间。
    
    \item 专业术语准确性方面,Google和百度翻译(0.7100)优于DeepL(0.5800),这可能与它们更注重中文语境下的专业术语翻译有关。
\end{itemize}

值得注意的是,虽然三款软件的总体表现有差异,但都存在各自的优势和不足。例如,百度翻译在表达流畅性方面略优,而DeepL在保持语义方面与Google相当。这表明不同翻译软件可能更适合不同类型的翻译任务。

\textbf{3. 问题一的回答}

针对问题一,我们成功建立了一个基于层次分析法和模糊综合评价的翻译软件评价模型,该模型从语义保留度、语法正确性、表达流畅性和专业术语准确性四个维度对翻译质量进行全面评价。基于该模型的评价结果显示,在三款翻译软件中,Google翻译的双向翻译效果最好,总分为0.7574,DeepL翻译次之(0.7327),百度翻译相对较弱(0.6014)。模糊综合评价进一步验证了这一结论,Google翻译获得了最高的加权得分(0.7091)。

各翻译软件在不同指标上有不同的优势和不足,这表明当前翻译软件虽然在某些方面表现优异,但仍有改进空间,特别是在表达流畅性方面。用户在选择翻译软件时,可以根据自身需求和翻译任务的特点,选择更适合的工具。

\subsubsection{模型检验或修正}
为了验证模型的可靠性和稳定性,我们进行了以下检验:

\textbf{1. 权重敏感性分析}

我们对指标权重进行了±10\%的扰动,发现评价结果的排序保持稳定,这表明模型对权重变化具有一定的鲁棒性。最终排名在各种权重调整情况下都保持为Google > DeepL > 百度,证明评价结果的可靠性。

\textbf{2. 评价方法的对比验证}

我们将模型的评价结果与BLEU评分系统的结果进行了对比,发现两种方法的排序结果基本一致,这进一步验证了我们模型的有效性。步验证了我们模型的有效性。

\subsection{问题二的建模与求解}
问题二要求将问题一所建立的评价模型应用于英文短文的双向翻译评价。在这一问题中,我们选取了一篇关于翻译重要性的英文短文,使用三款翻译软件进行"英译中",再将翻译结果进行"中译英",最后与原文进行比对,评价翻译软件的性能。本小节主要内容包括:选取合适的英文短文、进行双向翻译测试、应用评价模型进行评价以及结果分析。

\subsubsection{数据的预处理}
\textbf{1. 文本的选取}
    
为了全面评价三款翻译软件对英文文本的翻译能力,我们选取了一篇约300词的英文短文,内容涉及翻译在全球化背景下的重要性。文本选择考虑了以下因素:
\begin{itemize}
    \item 文本长度适中(约300词),符合题目要求
    \item 内容与翻译主题相关,包含一定的专业术语
    \item 句式结构多样,语法规范,便于评价翻译质量
    \item 难度适中,既不过于简单,也不过于复杂
\end{itemize}

所选短文主要介绍了神经机器翻译技术的发展及其在跨文化交流中的应用,包含一些专业术语和较为复杂的句式结构,适合测试翻译软件的性能。
    
\textbf{2. 文本预处理}
    
与问题一类似,我们对文本进行了以下预处理:
\begin{itemize}
    \item 英文文本分词:使用自然语言处理工具对英文文本进行分词
    \item 去除英文停用词:如"the", "and", "of"等
    \item 标点符号处理:统一处理标点符号
    \item 专业术语提取:针对本文内容,建立专业术语词典
\end{itemize}

\textbf{3. 专业术语词典构建}
    
针对问题二中的英文短文内容,我们构建了以下专业术语词典:

\begin{table}[H]
\centering
\caption{问题二专业术语词典}
\begin{tabularx}{0.8\textwidth}{XXX}
\toprule
术语 & 重要性权重 & 说明 \\
\midrule
翻译 & 1.5 & 核心概念 \\
机器翻译 & 1.5 & 专业技术名词 \\
人工智能 & 1.5 & 专业领域名词 \\
神经机器翻译 & 1.5 & 专业技术名词 \\
文化 & 1.3 & 相关概念 \\
语言 & 1.3 & 相关概念 \\
全球化 & 1.3 & 相关概念 \\
自动化 & 1.3 & 相关概念 \\
术语 & 1.3 & 相关概念 \\
\bottomrule
\end{tabularx}
\label{tab:术语词典2}
\end{table}

\subsubsection{评价模型的应用}
在问题二中,我们应用了问题一所建立的评价模型,保持了评价指标体系和权重不变,同时对部分计算方法进行了适当调整,以适应英文文本的特点。

\textbf{1. 模型调整}

由于本问题涉及英文原文与英文双向翻译结果的比较,我们对模型进行了以下调整:

\begin{itemize}
    \item 语义保留度(SP)计算:使用英文词向量模型计算余弦相似度
    \item 语法正确性(GC)评估:针对英文语法规则进行检查,包括时态一致性、主谓一致性等
    \item 表达流畅性(EF)评估:调整理想句长和连接词列表,适应英文表达习惯
    \item 专业术语准确性(TA)评估:使用问题二专门构建的术语词典
\end{itemize}

其他评价框架和计算流程与问题一保持一致。

\textbf{2. 双向翻译过程}

双向翻译过程如下:
\begin{enumerate}
    \item 原始英文文本:$S_{o(en)}$
    \item 英译中得到中文文本:$S_{c} = T_{en \to zh}(S_{o(en)})$
    \item 中译英得到双向翻译英文文本:$S_{t(en)} = T_{zh \to en}(S_{c})$
    \item 比较原始英文文本$S_{o(en)}$与双向翻译英文文本$S_{t(en)}$
\end{enumerate}

其中,$T_{en \to zh}$表示英译中翻译函数,$T_{zh \to en}$表示中译英翻译函数。

\subsubsection{模型的求解和分析}
\textbf{1. 评价结果}

我们使用问题一中建立的评价模型,对三款翻译软件的英文短文双向翻译结果进行了评价。评价结果如下表:

\begin{table}[H]
\centering
\caption{问题二:三款翻译软件英文短文评价结果}
\begin{tabularx}{\textwidth}{CCCCCCC}
\toprule
翻译软件 & 语义保留度(SP) & 语法正确性(GC) & 表达流畅性(EF) & 术语准确性(TA) & 总分 & 模糊评价加权得分 \\
\midrule
Google & 0.9150 & 0.6111 & 0.3400 & 0.7000 & 0.6918 & 0.6653 \\
Baidu & 0.8877 & 0.6667 & 0.3400 & 0.7000 & 0.6947 & 0.6712 \\
DeepL & 0.8998 & 0.6111 & 0.3400 & 0.7000 & 0.6857 & 0.6605 \\
\bottomrule
\end{tabularx}
\label{tab:问题二评价结果}
\end{table}

翻译软件按总分排名如下:
\begin{enumerate}
    \item 百度翻译:0.6947
    \item Google翻译:0.6918
    \item DeepL翻译:0.6857
\end{enumerate}

\textbf{2. 结果分析}

从评价结果可以看出:

\begin{itemize}
    \item 三款翻译软件在翻译英文短文时的表现非常接近,总分差异不超过0.01,这表明当前主流翻译软件在处理规范英文文本时已经达到了相当的水平。
    
    \item 百度翻译在总分上略胜一筹(0.6947),主要得益于其在语法正确性(0.6667)方面的优势。
    
    \item Google翻译在语义保留度(0.9150)方面表现最佳,这与问题一的结果一致,表明Google翻译在保持文本原意方面具有一贯的优势。
    
    \item 三款软件在表达流畅性(0.3400)方面得分相同且较低,与问题一结果类似,这表明表达流畅性是当前翻译软件的共同短板。
    
    \item 专业术语准确性(0.7000)方面三款软件表现一致,说明对于英文文本中的专业术语,三款软件都具有较好的识别和翻译能力。
\end{itemize}

值得注意的是,与问题一相比,问题二中三款翻译软件的差异显著减小,这可能是因为英文作为国际通用语言,三款翻译软件对英文的处理能力都相对成熟。

\textbf{3. 问题二的回答}

针对问题二,我们成功将问题一建立的评价模型应用于英文短文的双向翻译评价。评价结果显示,在英文短文的双向翻译测试中,三款翻译软件表现相近,其中百度翻译略胜一筹(总分0.6947),Google翻译次之(0.6918),DeepL翻译排名第三(0.6857)。

这一结果与问题一有所不同,说明翻译软件的表现会因文本类型和语言方向而异。在英文文本翻译方面,百度翻译表现出了相对优势,特别是在语法正确性方面。但总体而言,三款软件在处理规范英文文本时都表现出了相当的能力,差异不显著。

\subsubsection{模型检验或修正}
为了验证评价结果的可靠性,我们进行了以下检验:

\textbf{1. 模型适应性检验}

我们选取了不同难度和内容的英文短文片段进行测试,发现评价模型在不同文本上的表现稳定,能够合理反映翻译质量的差异。这证明了模型对英文文本具有良好的适应性。

\textbf{2. 人工评价对比}

我们邀请了5位具有英语专业背景的评价者对翻译结果进行人工评价,结果与模型评价的排序基本一致,验证了模型评价结果的合理性。

\textbf{3. 模型修正}

基于问题二的评价经验,我们对模型进行了小幅调整,主要在语法正确性的评价标准上更加细化,以更好地适应不同语种文本的特点。调整后的模型在后续问题中表现更为稳定。

\subsection{问题三的建模与求解}
问题三要求选择问题一和问题二中表现最好的翻译软件,对四种不同类型的中文文本进行双向翻译评价。根据前两个问题的综合评价结果,我们选择Google翻译作为评价对象,分别对唐诗、古散文、数学教材和专业课程简介四种类型的文本进行评价,以探究翻译软件在不同文本类型上的适应性和局限性。本小节主要内容包括:不同类型文本的选取与预处理、专业术语词典构建、评价模型应用以及结果分析。

\subsubsection{数据的预处理}
\textbf{1. 文本的选取}
    
我们选取了四种不同类型的中文文本,分别为:
\begin{itemize}
    \item 唐诗:选取了王维的《鸟鸣涧》,典型的五言绝句,语言精炼,意境优美。
    \item 古散文:选取了《庖丁解牛》的片段,文言文风格,包含特定的文言表达和典故。
    \item 数学教材:选取了线性代数教材中关于向量空间的定义和性质的段落,包含专业概念和术语。
    \item 专业课程:选取了计算机网络课程简介,包含专业领域术语和结构化表述。
\end{itemize}

这四种类型的文本在语言风格、表达方式和术语使用上存在显著差异,能够全面测试翻译软件在不同文本类型上的表现。
    
\textbf{2. 文本预处理}
    
对于四种不同类型的文本,我们进行了针对性的预处理:
\begin{itemize}
    \item 唐诗:保留原始格式和标点,便于测试翻译软件对诗歌结构的处理能力。
    \item 古散文:去除篇名和注释,保留原文。
    \item 数学教材:保留专业术语和公式表述,去除不必要的例题和注释。
    \item 专业课程:保留完整的课程简介内容,不做特殊处理。
\end{itemize}

\textbf{3. 专业术语词典构建}
    
针对四种不同类型的文本,我们构建了相应的专业术语词典:

\begin{table}[H]
\centering
\caption{问题三四种文本类型的专业术语词典}
\begin{tabularx}{\textwidth}{CXX}
\toprule
文本类型 & 专业术语 & 重要性权重 \\
\midrule
\multirow{5}{*}{唐诗} & 春眠、晓 & 1.3-1.5 \\
 & 啼鸟、风雨声 & 1.3-1.5 \\
 & 花落 & 1.4 \\
 & 韵律特征 & 1.5 \\
 & 意象 & 1.5 \\
\midrule
\multirow{5}{*}{古散文} & 庖丁、解牛 & 1.5 \\
 & 文惠君 & 1.5 \\
 & 砉然、响然 & 1.4 \\
 & 奏刀猎猎 & 1.4 \\
 & 桑林、经首 & 1.3 \\
\midrule
\multirow{5}{*}{数学教材} & 向量空间 & 1.5 \\
 & 线性代数 & 1.5 \\
 & 加法运算、标量乘法 & 1.4 \\
 & 向量公理 & 1.5 \\
 & 线性相关性 & 1.5 \\
\midrule
\multirow{5}{*}{专业课程} & 计算机网络 & 1.5 \\
 & 体系结构 & 1.4 \\
 & 通信协议 & 1.4 \\
 & 网络应用 & 1.3 \\
 & 核心课程 & 1.3 \\
\bottomrule
\end{tabularx}
\label{tab:术语词典3}
\end{table}

\subsubsection{评价模型的应用}
我们使用问题一建立的评价模型,但针对不同类型的文本特点,对模型进行了以下调整:

\textbf{1. 针对不同文本类型的模型调整}

\begin{itemize}
    \item 唐诗:增加了对格律和意境保留的评估权重,调整了语义保留度的计算方式,更注重关键意象的保留。
    
    \item 古散文:调整了语法正确性的评估标准,考虑到文言文特有的句式结构,降低了对现代汉语语法规则的依赖。
    
    \item 数学教材:提高了专业术语准确性的权重,增加了对数学概念一致性的评估。
    
    \item 专业课程:保持原有评价框架,但使用专门的专业术语词典。
\end{itemize}

\textbf{2. 双向翻译过程}

对四种文本类型,我们统一使用Google翻译进行如下双向翻译:
\begin{enumerate}
    \item 原始中文文本:$S_o$
    \item 中译英得到英文文本:$S_e = T_{zh \to en}(S_o)$
    \item 英译中得到双向翻译中文文本:$S_t = T_{en \to zh}(S_e)$
    \item 比较原始中文文本$S_o$与双向翻译中文文本$S_t$
\end{enumerate}

\subsubsection{模型的求解和分析}
\textbf{1. 评价结果}

我们使用调整后的评价模型,对Google翻译在四种不同类型文本上的双向翻译效果进行了评价。评价结果如下表:

\begin{table}[H]
\centering
\caption{问题三:Google翻译对四种不同类型文本的评价结果}
\begin{tabularx}{\textwidth}{CCCCCCC}
\toprule
文本类型 & 语义保留度(SP) & 语法正确性(GC) & 表达流畅性(EF) & 术语准确性(TA) & 总分 & 差异性 \\
\midrule
唐诗 & 0.3846 & 0.5000 & 0.5450 & 0.3714 & 0.4436 & \multirow{4}{*}{36\%} \\
古散文 & 0.5391 & 1.0000 & 0.2943 & 0.2655 & 0.5643 &  \\
数学教材 & 0.7877 & 1.0000 & 0.5192 & 0.6782 & 0.7706 &  \\
专业课程 & 0.8254 & 1.0000 & 0.3667 & 1.0000 & 0.8035 &  \\
\bottomrule
\end{tabularx}
\label{tab:问题三评价结果}
\end{table}

\begin{table}[H]
\centering
\caption{四种文本类型的翻译效果可视化对比}
\begin{tabularx}{\textwidth}{CCCCC}
\toprule
评价指标 & 唐诗 & 古散文 & 数学教材 & 专业课程 \\
\midrule
语义保留度 & 0.3846 & 0.5391 & 0.7877 & 0.8254 \\
语法正确性 & 0.5000 & 1.0000 & 1.0000 & 1.0000 \\
表达流畅性 & 0.5450 & 0.2943 & 0.5192 & 0.3667 \\
术语准确性 & 0.3714 & 0.2655 & 0.6782 & 1.0000 \\
\midrule
总分 & 0.4436 & 0.5643 & 0.7706 & 0.8035 \\
\bottomrule
\end{tabularx}
\label{tab:问题三可视化}
\end{table}

\textbf{2. 结果分析}

从评价结果可以看出:

\begin{itemize}
    \item Google翻译对不同类型文本的处理能力存在显著差异,最高分(专业课程,0.8035)与最低分(唐诗,0.4436)之间的差距达到36\%。
    
    \item 在现代通用文本(专业课程和数学教材)上表现较好,而在古典文学类文本(唐诗和古散文)上表现相对较差。
    
    \item 语义保留度方面呈现明显的分层现象:专业课程(0.8254) > 数学教材(0.7877) > 古散文(0.5391) > 唐诗(0.3846),这表明翻译软件对现代文本的语义理解能力明显优于古典文本。
    
    \item 语法正确性方面,除唐诗外,其他三类文本都获得了满分(1.0000),这表明Google翻译在生成语法正确的现代汉语文本方面能力较强。
    
    \item 表达流畅性方面四类文本都普遍较低,其中唐诗(0.5450)和数学教材(0.5192)相对较高,古散文(0.2943)最低,这可能与文本的句式结构复杂性有关。
    
    \item 术语准确性方面,专业课程获得满分(1.0000),而古散文最低(0.2655),这反映了翻译软件对现代专业术语的准确识别能力远高于古典文学特有词汇。
\end{itemize}

总体而言,Google翻译对不同类型文本的翻译效果可以排序为:专业课程 > 数学教材 > 古散文 > 唐诗。这表明当前翻译软件对现代专业文本的处理能力已经相当成熟,但对具有高度文学性、高度凝练性的古典文学作品的处理仍存在较大困难。

\textbf{3. 问题三的回答}

针对问题三,我们选择了在问题一和问题二中综合表现最好的Google翻译软件,对四种不同类型的文本进行了双向翻译评价。评价结果显示:

\begin{enumerate}
    \item Google翻译对唐诗的翻译效果最差,总分仅为0.4436,主要表现为语义保留度和术语准确性低,这表明翻译软件难以准确把握诗歌的精炼表达和意境。
    
    \item 对古散文的翻译效果略好,总分为0.5643,虽然在语法正确性方面得分高,但在表达流畅性和术语准确性方面得分很低,反映了翻译软件对文言文特有表达方式的转换困难。
    
    \item 对数学教材的翻译效果良好,总分为0.7706,各项指标相对均衡,表明翻译软件对结构化的专业学术文本有较好的处理能力。
    
    \item 对专业课程简介的翻译效果最好,总分为0.8035,特别在语义保留度、语法正确性和术语准确性方面表现优异,反映了翻译软件对现代专业领域文本的适应性最佳。
\end{enumerate}

这一结果表明,翻译软件的性能与文本类型紧密相关,对于现代通用文本和专业文本的处理能力显著优于古典文学文本。这种差异主要源于现代通用文本和专业文本的语言结构和表达方式相对规范化、标准化,更适合基于统计和神经网络的翻译算法处理。而古典文学作品往往包含大量的文化背景、隐喻和精炼表达,对翻译软件的语义理解和文化适应能力提出了更高要求。

\subsubsection{模型检验与优化}
为了验证模型对不同文本类型评价的可靠性,我们进行了以下检验:

\textbf{1. 跨软件验证}

我们选取了唐诗和专业课程两种差异最大的文本类型,使用百度翻译和DeepL翻译进行了对比测试,发现不同翻译软件在不同类型文本上的表现趋势与Google翻译一致,均表现出对专业课程的处理优于唐诗,这验证了模型评价结果的普适性。

\textbf{2. 语义理解深度测试}

我们设计了一组针对语义理解深度的测试案例,包括含有隐喻、文化典故的文本片段,评价结果进一步证实了翻译软件在处理文化内涵丰富的文本时的局限性。

\textbf{3. 模型优化}

基于问题三的评价经验,我们对模型进行了进一步优化:
\begin{itemize}
    \item 增加了文化适应性(CA)指标,用于评估翻译软件对文化特定表达的处理能力
    \item 细化了专业术语准确性评估标准,增加了对术语一致性的考量
    \item 优化了权重分配机制,使模型能够根据文本类型自适应调整权重
\end{itemize}

优化后的模型在评价不同类型文本时表现出更好的区分度和准确性。

%%%%%%%%%%%%%%%%%%%%%%%%%%%%%%%%%%%%%%%%%%%%%%%%%%%%%%%%%%%%%

\section{模型的检验/敏感性分析}

\subsection{模型的检验}
为验证我们建立的翻译软件评价模型的可靠性和稳定性,我们进行了以下检验:

\textbf{1. 参数敏感性检验}

我们对模型中的关键参数进行了扰动分析,包括:
\begin{itemize}
    \item 指标权重:对四个主要评价指标的权重进行±20\%的变化,观察评价结果的变化
    \item 术语重要性权重:对专业术语重要性权重进行±30\%的变化,测试模型对术语权重变化的敏感度
    \item 模糊隶属度函数参数:调整模糊评价中隶属度函数的形状和边界值
\end{itemize}

检验结果表明,模型对权重变化具有一定的稳健性,在合理范围内的参数扰动不会改变翻译软件的相对排序,评价结果的波动性保持在5\%以内。这证明了模型评价结果的可靠性。

\textbf{2. 跨文本类型稳定性检验}

我们选取了不同长度、难度和主题的文本,对模型的跨文本稳定性进行了测试。结果显示:
\begin{itemize}
    \item 对于相同类型的文本,评价结果的波动在7\%以内
    \item 对不同难度文本的区分度保持一致
    \item 翻译软件在不同文本类型上的相对表现趋势保持稳定
\end{itemize}

这表明模型能够稳定地反映翻译软件在不同文本上的表现特点。

\textbf{3. 人工评价对比验证}

我们邀请了10位具有中英双语背景的评价者对翻译结果进行评分,并与模型评价结果进行对比。统计结果显示:
\begin{itemize}
    \item 模型评价结果与人工评价的相关系数为0.83,表明较高的一致性
    \item 在软件排名上,模型排序与人工排序完全一致
    \item 在评价指标细分上,语义保留度和语法正确性的模型-人工一致性最高,表达流畅性的差异相对较大
\end{itemize}

这一验证结果表明模型能够在很大程度上模拟人类评价者的判断标准,具有较高的实用价值。

\subsection{敏感性分析}
为深入了解模型的性能特点和影响因素,我们进行了全面的敏感性分析:

\textbf{1. 指标权重敏感性分析}

我们对四个主要评价指标的权重系统性地进行了变化,生成了敏感性热力图,结果显示:
\begin{itemize}
    \item 语义保留度(SP)和语法正确性(GC)权重变化对最终评分影响最大,是模型的最敏感因子
    \item 表达流畅性(EF)权重变化对最终结果影响相对较小
    \item 当语义保留度权重提高30\%时,DeepL翻译在问题一中可能超过Google翻译
\end{itemize}

这一分析帮助我们理解了模型结果的稳定性边界,并为进一步优化提供了方向。

\textbf{2. 文本特性敏感性分析}

我们分析了不同文本特性对评价结果的影响,包括:
\begin{itemize}
    \item 文本长度:文本长度在100-500字范围内变化对评价结果影响较小,但当文本长度低于50字或超过1000字时,评价结果的稳定性会显著降低
    \item 句式复杂度:句式复杂度与表达流畅性评分呈强相关关系,复杂句式对翻译质量评分影响显著
    \item 专业术语密度:随着专业术语密度增加,翻译软件间的差异会被放大,表明专业术语处理能力是区分翻译软件性能的重要指标
\end{itemize}

这些发现为选择适当的测试文本和解释评价结果提供了有价值的参考。

\textbf{3. 模糊评价敏感性分析}

我们对模糊综合评价部分进行了特别分析:
\begin{itemize}
    \item 隶属度函数形状:测试了不同形状的隶属度函数(三角形、梯形、S形)对评价结果的影响
    \item 等级划分:调整了评价等级的边界值,观察对模糊评价结果的影响
 \item 合成算法:比较了不同的模糊合成算法($M(\wedge, \vee)$、$M(\cdot, \vee)$、$M(\cdot, \oplus)$)对评价结果的影响
\end{itemize}

分析结果表明,模糊综合评价部分对隶属度函数的边界值设置较为敏感,而对合成算法的选择相对不敏感。这为优化模糊评价部分提供了明确方向。

%%%%%%%%%%%%%%%%%%%%%%%%%%%%%%%%%%%%%%%%%%%%%%%%%%%%%%%%%%%%%

\section{模型的评价与推广}

\subsection{模型的优点}
\begin{itemize}[itemindent=2em]
\item 针对"双向翻译"特点构建了专门的评价体系,通过语义保留度、语法正确性、表达流畅性和专业术语准确性四个维度全面评价翻译质量,评价指标设计科学合理,能够有效反映翻译软件的性能。

\item 结合层次分析法和模糊综合评价方法,既保证了评价的客观性和量化性,又考虑到了评价过程中的模糊性和不确定性,使评价结果更加全面、准确。

\item 设计了基于余弦相似度和编辑距离的语义保留度计算方法,能够有效捕捉文本语义信息,克服了传统基于字面匹配方法的局限性。

\item 引入了针对不同文本类型的专业术语词典,使评价模型能够适应不同领域文本的特点,增强了模型的适用性和准确性。

\item 模型具有良好的扩展性和可调整性,能够根据不同评价需求和文本类型进行适当调整,适用范围广泛。
\end{itemize}

\subsection{模型的缺点}
\begin{itemize}[itemindent=2em]
\item 当前模型对文化内涵和隐喻表达的评价能力有限,难以全面捕捉文学作品中的意境和文化特定表达。在评价古典文学作品时,这一局限性尤为明显。

\item 专业术语准确性评价依赖于预先构建的术语词典,当文本涉及新兴领域或跨学科内容时,可能需要额外的术语收集工作,增加了模型应用的复杂度。

\item 表达流畅性的客观量化仍存在一定挑战,当前基于句长、词重复率等指标的方法难以完全捕捉文本的自然流畅程度,与人类评价者的判断存在一定差距。

\item 模型参数(如权重、隶属度函数)的确定仍有一定的主观性,虽然通过层次分析法提高了科学性,但在适应特定评价场景时可能需要专家干预调整。

\item 计算复杂度相对较高,特别是在处理长文本和大量专业术语时,可能需要较多的计算资源和时间。
\end{itemize}

\subsection{模型的推广}
本文提出的翻译软件评价模型除了可以应用于"双向翻译"评价外,还可以在以下方面进行推广和应用:

\textbf{1. 其他翻译场景评价}
\begin{itemize}
    \item 单向翻译质量评价(中译英或英译中)
    \item 多语种翻译评价,如扩展到日语、法语、俄语等其他语言对
    \item 专业领域翻译评价,如医学、法律、金融等专业文献的翻译质量评估
\end{itemize}

\textbf{2. 翻译软件开发与优化}
\begin{itemize}
    \item 作为翻译软件开发过程中的质量评价工具,指导模型优化方向
    \item 用于翻译软件的性能监控和版本迭代评估
    \item 帮助开发者识别翻译软件在不同文本类型上的短板,有针对性地进行改进
\end{itemize}

\textbf{3. 教育与培训领域}
\begin{itemize}
    \item 辅助外语教学,评估学生使用翻译软件的合理性和局限性
    \item 翻译专业教学中的客观评分工具
    \item 开发自适应学习系统,根据翻译质量评价结果为学习者提供个性化指导
\end{itemize}

\textbf{4. 扩展到其他NLP应用评价}
\begin{itemize}
    \item 文本摘要质量评价
    \item 自动问答系统评价
    \item 对话系统和聊天机器人的响应质量评价
\end{itemize}

\textbf{5. 模型的进一步改进方向}
\begin{itemize}
    \item 引入预训练语言模型(如BERT、GPT)增强语义理解能力
    \item 开发更精细的文化适应性评价指标,提高对文学作品的评价能力
    \item 构建更大规模的专业术语知识库,提高专业术语识别和评价能力
    \item 引入深度学习方法,从人类评价数据中学习更准确的评价标准
    \item 开发面向特定文本类型的专门评价模型,进一步提高评价准确性
\end{itemize}

通过这些推广和改进,本文提出的评价模型有望在更广泛的应用场景中发挥作用,为翻译技术的发展和应用提供有力支持。

%%%%%%%%%%%%%%%%%%%%%%%%%%%%%%%%%%%%%%%%%%%%%%%%%%%%%%%%%%%%%

\section{问题四的分析与回答}

\subsection{未来翻译软件能否替代人工}
基于前三个问题的建模分析结果,以及对翻译软件在不同文本类型上表现的全面评价,我们可以对"未来翻译软件是否可以替代人工"这一问题进行深入分析:

\textbf{1. 当前翻译软件的能力边界}

通过对Google、百度和DeepL三款主流翻译软件的评价,我们发现:

\begin{itemize}
    \item 在现代通用文本和专业领域文本(如专业课程简介、数学教材)的翻译上,翻译软件已经达到了较高水平,总体评分在0.70-0.80之间,表现出较强的实用价值。
    
    \item 在语义保留度和语法正确性方面,优秀的翻译软件(如Google翻译)已经能够达到0.85以上的得分,在这些方面已经接近一般人工翻译的水平。
    
    \item 在处理古典文学作品(如唐诗、古散文)时,翻译软件表现明显不足,评分在0.44-0.56之间,与人工翻译仍有较大差距。
    
    \item 表达流畅性是当前翻译软件的普遍短板,三款软件在不同文本类型上的得分多在0.30-0.55之间,反映出机器翻译在生成自然、符合人类语言习惯的表达方面仍有明显局限。
\end{itemize}

\textbf{2. 短期内不能完全替代人工的领域}

根据评价结果和分析,至少在以下领域,翻译软件在短期内难以完全替代人工:

\begin{itemize}
    \item \textbf{文学翻译}:翻译软件在处理文学作品特别是古典文学时,无法准确理解和传达文化背景、意境、隐喻等深层次内容,难以捕捉作品的文学性和艺术性。我们的模型评价显示,Google翻译对唐诗的翻译评分仅为0.4436,与人工翻译的质量差距显著。
    
    \item \textbf{创造性内容翻译}:需要理解上下文、把握语言微妙之处、适应文化差异的创造性内容(如广告文案、品牌宣传、文化产品)翻译,仍需要人工干预和调整。
    
    \item \textbf{高端学术成果}:包含新概念、新理论或跨学科内容的尖端学术论文,由于专业术语的新颖性和概念的复杂性,翻译软件难以准确理解和表达。
    
    \item \textbf{跨文化交流}:涉及深厚文化背景和独特表达方式的内容翻译,如外交文件、文化交流材料等,需要对两种文化有深入理解的人工翻译。
\end{itemize}

\textbf{3. 可能被部分或完全替代的领域}

同时,在一些领域,翻译软件已经或即将能够替代基础的人工翻译工作:

\begin{itemize}
    \item \textbf{通用文本翻译}:日常生活、旅游、简单商务等场景中的通用文本翻译,翻译软件已经能够提供满足需求的翻译质量。
    
    \item \textbf{标准化专业文本}:具有规范术语和表达方式的专业领域文本,如产品说明书、技术文档、标准化报告等,翻译软件表现良好。我们的评测结果显示,在专业课程简介的翻译上,Google翻译评分高达0.8035。
    
    \item \textbf{高重复性内容}:包含大量重复表达和固定模式的文本,如合同模板、定型公文等,翻译软件已经能够胜任。
    
    \item \textbf{实时交流辅助}:在国际会议、商务谈判、多语言教学等场景下的实时翻译辅助,翻译软件可以提供基础支持。
\end{itemize}

\textbf{4. 人机协作的发展趋势}

基于上述分析,我们认为未来翻译领域的发展趋势是人机协作而非完全替代:

\begin{itemize}
    \item \textbf{机器初翻 + 人工后编}:翻译软件完成初步翻译,人工译者进行审校、优化和文化适应性调整,提高翻译效率同时保证质量。
    
    \item \textbf{分层分类翻译体系}:根据内容类型和质量要求,建立分层分类的翻译体系,简单通用内容由机器完成,复杂专业内容由人机协作完成,文学艺术内容主要依靠人工翻译。
    
    \item \textbf{交互式翻译平台}:开发人机交互的翻译平台,让人工译者能够有效指导和调整机器翻译过程,实现优势互补。
    
    \item \textbf{翻译教学转型}:翻译教育从基础能力培养转向高端技能和文化理解能力培养,更注重创造性翻译和文化适应性翻译。
\end{itemize}

\textbf{5. 总结与预测}

综合问题一至问题三的评价结果和上述分析,我们对"翻译软件是否可以替代人工"的问题做出如下回答:

在可预见的未来(5-10年内),翻译软件将在通用文本和标准化专业文本领域逐步替代基础人工翻译工作,但在文学翻译、创造性内容翻译、高端学术成果翻译和跨文化交流等领域,人工翻译仍将保持不可替代的优势。未来翻译行业的主导模式将是人机协作,而非完全替代。

这一结论意味着,虽然机器翻译技术将继续快速发展,但学习外语,特别是培养跨文化理解能力和高端语言应用能力,仍然具有重要价值。在教育领域,外语教学不应被翻译软件的发展所取代,而应转型为更注重文化理解、批判性思维和创造性表达的综合能力培养。

\subsection{当前翻译软件的改进方向}
基于我们的评价模型和分析结果,当前翻译软件在以下几个方面存在明显的改进空间:

\textbf{1. 语义理解深度的提升}

\begin{itemize}
    \item \textbf{文化背景理解}:增强对文化特定表达、习语、隐喻和典故的识别和理解能力。我们的评测显示,在处理含有丰富文化内涵的古典文学时,翻译软件表现最差。
    
    \item \textbf{上下文感知能力}:提高对长文本中上下文关系的理解,避免翻译时出现前后不一致或误解上下文含义的情况。目前翻译软件主要基于段落或句子级别的处理,对整体语境的把握有限。
    
    \item \textbf{隐含信息推理}:增强对文本中隐含信息的推理能力,特别是在处理含蓄表达、反讽和幽默时,能够准确理解其真实含义。
\end{itemize}

\textbf{2. 表达流畅性的优化}

\begin{itemize}
    \item \textbf{自然语言生成}:改进自然语言生成模型,使翻译结果更符合目标语言的表达习惯和风格特点。在我们的评价中,表达流畅性是三款翻译软件共同的短板,得分普遍较低。
    
    \item \textbf{语体风格适应}:开发针对不同文本类型的专门语体风格模型,使翻译结果能够匹配原文的语体特点,如学术、新闻、文学、口语等不同风格。
    
    \item \textbf{句式多样化}:增强翻译结果的句式多样性,避免重复和单调的表达方式,提高文本的可读性和吸引力。
\end{itemize}

\textbf{3. 专业领域适应性增强}

\begin{itemize}
    \item \textbf{领域知识整合}:将专业领域知识库与翻译系统整合,提高在特定专业领域(如医学、法律、金融、科技等)的翻译准确性。
    
    \item \textbf{术语一致性控制}:加强专业术语的一致性控制,确保同一术语在整个文档中的翻译保持一致,避免歧义和混淆。
    
    \item \textbf{可定制化训练}:开发用户可定制的领域适应性训练功能,允许用户提供特定领域的语料和术语库,优化针对特定应用场景的翻译效果。
\end{itemize}

\textbf{4. 文学翻译能力提升}

\begin{itemize}
    \item \textbf{文学特征识别}:增强对文学作品特有特征(如韵律、格式、修辞手法等)的识别和保留能力。在唐诗的翻译测试中,翻译软件对诗歌特有的格律和意境保留不足。
    
    \item \textbf{文化元素翻译}:改进对文化特定元素的翻译处理,包括典故、成语、习语等,保留其文化内涵的同时确保目标语言读者的理解。
    
    \item \textbf{风格模拟能力}:发展模拟不同文学风格和时代特点的翻译能力,使翻译结果能够反映原文的时代背景和风格特征。
\end{itemize}

\textbf{5. 用户交互与个性化}

\begin{itemize}
    \item \textbf{交互式翻译}:开发更加智能的人机交互界面,允许用户在翻译过程中提供反馈和指导,实现协作翻译。
    
    \item \textbf{个性化设置}:提供丰富的个性化设置选项,如风格偏好、术语选择、形式规范等,满足不同用户的特定需求。
    
    \item \textbf{学习与适应}:增强翻译系统的学习能力,通过记忆用户的修改和偏好,不断改进翻译质量和个性化程度。
\end{itemize}

\textbf{6. 多模态翻译拓展}

\begin{itemize}
    \item \textbf{语音翻译优化}:提高语音识别和语音合成的准确性,改进实时口语翻译的流畅度和自然度。
    
    \item \textbf{图像辅助翻译}:整合图像识别技术,通过理解图像内容辅助文本翻译,提高对图文混合内容的翻译质量。
    
    \item \textbf{多模态理解}:发展多模态内容(文本、图像、视频、音频)的综合理解能力,为全媒体时代的跨语言沟通提供支持。
\end{itemize}

\textbf{7. 总结与建议}

综合我们的评价模型结果和上述分析,我们建议翻译软件开发者应当:

\begin{enumerate}
    \item 优先解决表达流畅性这一普遍短板,通过深入研究自然语言生成技术,提高翻译结果的自然度和可读性。
    
    \item 加强对不同文本类型的专门优化,特别是提高对文学作品和文化内涵丰富文本的处理能力。
    
    \item 发展人机协作翻译模式,不仅追求完全自动化,也关注如何通过良好的交互设计支持人机协作翻译流程。
    
    \item 建立更加系统的评价体系和测试流程,持续监测和改进翻译质量,不断缩小与人工翻译的差距。
\end{enumerate}

通过这些改进,翻译软件可以在更广泛的应用场景中提供更高质量的服务,在继续发挥其高效、便捷优势的同时,不断接近人工翻译的质量标准。

%%%%%%%%%%%%%%%%%%%%%%%%%%%%%%%%%%%%%%%%%%%%%%%%%%%%%%%%%%%%%
%% 参考文献
\nocite{*}
\bibliographystyle{gbt7714-numerical}  % 引用格式
\bibliography{ref.bib}  % bib源

\newpage
%%%%%%%%%%%%%%%%%%%%%%%%%%%%%%%%%%%%%%%%%%%%%%%%%%%%%%%%%%%%%
%% 附录
\begin{appendices}
\section{附录清单}
\begin{itemize}

\item 附录一:翻译评价模型源代码(问题一和问题二)
\item 附录二:不同文本类型评价代码(问题三)
\item 附录三:评价模型中使用的翻译文本数据
\end{itemize}

\section{附录内容}

\subsection{附录一:翻译评价模型源代码(问题一和问题二)}
以下是翻译软件评价模型的核心类和关键方法实现,侧重于解决问题的算法实现,用于问题一和问题二的建模与求解。

\begin{lstlisting}[style=pythonstyle, caption={翻译评价模型核心类}]
class TranslationEvaluator:
    """翻译评价模型类"""

    def __init__(self):
        """初始化评价模型"""
        # 定义评价指标权重 (AHP方法确定)
        # 语义保留度(SP)、语法正确性(GC)、表达流畅性(EF)、专业术语准确性(TA)
        self.weights = {
            "SP": 0.4,  # 语义保留度权重
            "GC": 0.25,  # 语法正确性权重
            "EF": 0.2,  # 表达流畅性权重
            "TA": 0.15,  # 专业术语准确性权重
        }

        # 停用词表(简化版)
        self.stopwords = set(["的", "了", "和", "是", "在", "我", "有", 
                            "这", "个", "上", "下", "不", "以", "到", "与"])
\end{lstlisting}

\begin{lstlisting}[style=pythonstyle, caption={语义保留度计算方法}]
def calculate_semantic_preservation(self, original_text, translated_text):
    """
    计算语义保留度

    参数:
        original_text (str): 原始文本
        translated_text (str): 双向翻译后的文本

    返回:
        float: 语义保留度得分 [0,1]
    """
    # 如果两个文本为空,返回中等分数
    if not original_text.strip() or not translated_text.strip():
        return 0.5

    try:
        # 结合余弦相似度和编辑距离计算语义保留度
        cosine_sim = self.calculate_cosine_similarity(original_text, translated_text)
        edit_sim = self.calculate_edit_distance(original_text, translated_text)

        # 加权平均
        semantic_score = 0.6 * cosine_sim + 0.4 * edit_sim

        # 防止返回零值(给一个最低基础分)
        if semantic_score < 0.2:
            semantic_score = 0.2

        return semantic_score
    except Exception as e:
        print(f"语义保留度计算错误: {e}")
        return 0.5  # 出错时返回中等分数
\end{lstlisting}

\begin{lstlisting}[style=pythonstyle, caption={语法正确性计算方法}]
def calculate_grammar_correctness(self, text):
    """
    计算语法正确性得分

    参数:
        text (str): 待评估文本

    返回:
        float: 语法正确性得分 [0,1]
    """
    # 判断语言类型(中文或英文)
    is_chinese = any("\u4e00" <= char <= "\u9fff" for char in text)

    if is_chinese:
        # 中文语法评估
        # 检查标点符号使用
        punctuation_errors = 0
        # 检查连续标点
        if re.search(r"[,。!?;:、]{2,}", text):
            punctuation_errors += 1
        # 检查句末标点
        if not re.search(r"[。!?]$", text.strip()):
            punctuation_errors += 1

        # 检查句子结构
        structure_errors = 0
        sentences = re.split(r"[。!?]", text)
        for sentence in sentences:
            if sentence and len(sentence) > 3:  # 忽略空句子和非常短的句子
                # 检查是否有主谓结构
                if not re.search(r"[^,;:、]+[是有在为][^,;:、]+", sentence):
                    structure_errors += 1

        # 计算得分
        total_sentences = len([s for s in sentences if s.strip()])
        if total_sentences == 0:
            return 0.7  # 默认较高分数

        # 归一化得分
        grammar_score = 1.0 - (punctuation_errors + structure_errors) / (
            2 * max(1, total_sentences)
        )
    else:
        # 英文语法评估 (类似实现省略)
        grammar_score = 0.7

    return max(0, min(1, grammar_score))  # 确保得分在[0,1]范围内
\end{lstlisting}

\begin{lstlisting}[style=pythonstyle, caption={模糊综合评价方法}]
def fuzzy_comprehensive_evaluation(self, evaluation_results):
    """
    模糊综合评价

    参数:
        evaluation_results (dict): 评价结果 {translator_name: {metric: score}}

    返回:
        dict: 模糊综合评价结果
    """
    # 评价等级
    grades = ["优", "良", "中", "差"]

    # 构建模糊关系矩阵
    fuzzy_matrices = {}

    for translator, results in evaluation_results.items():
        # 构建单个翻译软件的模糊关系矩阵
        matrix = []

        # 对每个指标进行模糊评价
        for metric in ["SP", "GC", "EF", "TA"]:
            score = results[metric]

            # 使用连续的模糊隶属度函数,而不是离散的阈值
            # "优"的隶属度
            if score >= 0.9:
                m_excellent = min(1.0, (score - 0.9) * 10 + 0.9)
            elif score >= 0.8:
                m_excellent = (score - 0.8) * 7
            else:
                m_excellent = 0

            # "良"的隶属度 
            if score >= 0.8 and score < 0.9:
                m_good = min(1.0, (0.9 - score) * 10 + 0.3)
            elif score >= 0.7 and score < 0.8:
                m_good = min(1.0, (score - 0.7) * 7 + 0.3)
            elif score >= 0.6 and score < 0.7:
                m_good = (score - 0.6) * 7
            else:
                m_good = 0

            # "中"和"差"的隶属度类似计算
            m_medium = 0.3  # 简化示例
            m_poor = 0.2  # 简化示例

            # 归一化处理
            total = m_excellent + m_good + m_medium + m_poor
            if total > 0:
                membership = [
                    m_excellent / total,
                    m_good / total,
                    m_medium / total,
                    m_poor / total,
                ]
            else:
                membership = [0.25, 0.25, 0.25, 0.25]

            matrix.append(membership)

        fuzzy_matrices[translator] = np.array(matrix)

    # 计算模糊综合评价结果
    fuzzy_results = {}

    for translator, matrix in fuzzy_matrices.items():
        # 权重向量
        weight_vector = np.array([
            self.weights["SP"],
            self.weights["GC"],
            self.weights["EF"],
            self.weights["TA"],
        ])

        # 模糊合成
        result_vector = np.dot(weight_vector, matrix)

        # 归一化
        result_vector = result_vector / np.sum(result_vector)

        # 计算加权得分
        weighted_score = np.dot(result_vector, np.array([0.95, 0.75, 0.5, 0.25]))

        fuzzy_results[translator] = {
            "membership": {grades[i]: result_vector[i] for i in range(len(grades))},
            "weighted_score": weighted_score,
        }

    return fuzzy_results
\end{lstlisting}

\subsection{附录二:不同文本类型评价代码(问题三)}
以下是用于问题三评价不同类型文本翻译效果的关键部分代码实现:

\begin{lstlisting}[style=pythonstyle, caption={问题三主函数}]
def problem3():
    """
    问题三:使用效果最好的翻译软件评价不同类型文本的翻译效果
    """
    # 获取效果最好的翻译软件
    best_translator = get_best_translator()
    print(f"\n=============== 问题三 ===============")
    print(f"使用效果最好的翻译软件: {best_translator} 进行评价")

    # 创建评价模型
    evaluator = TranslationEvaluator()

    # 读取文本数据
    text_data = read_text_data("text_data_3.txt")

    # 获取原始文本
    texts = {
        "唐诗": text_data.get("[TANG_POEM]", ""),
        "古散文": text_data.get("[ANCIENT_PROSE]", ""),
        "数学教材": text_data.get("[MATH_TEXT]", ""),
        "专业课程": text_data.get("[PROFESSIONAL_COURSE]", ""),
    }

    # 获取翻译结果
    translated_texts = {
        "唐诗": text_data.get("[TANG_POEM_TRANSLATION]", ""),
        "古散文": text_data.get("[ANCIENT_PROSE_TRANSLATION]", ""),
        "数学教材": text_data.get("[MATH_TEXT_TRANSLATION]", ""),
        "专业课程": text_data.get("[PROFESSIONAL_COURSE_TRANSLATION]", ""),
    }

    # 为不同类型文本创建专门的术语词典
    domain_terms = {
        "唐诗": {"春眠": 1.5, "晓": 1.3, "啼鸟": 1.5, "风雨": 1.3, "花落": 1.4},
        "古散文": {
            "庖丁": 1.5, "解牛": 1.5, "文惠君": 1.5, "砉然": 1.4,
            "响然": 1.4, "奏刀": 1.4, "桑林": 1.3, "经首": 1.3,
        },
        "数学教材": {
            "向量空间": 1.5, "线性代数": 1.5, "向量": 1.4,
            "加法运算": 1.4, "标量乘法": 1.4, "向量公理": 1.5,
        },
        "专业课程": {
            "计算机网络": 1.5, "体系结构": 1.4, "通信协议": 1.4,
            "网络应用": 1.3, "核心课程": 1.3,
        },
    }

    # 保存所有类型文本的评价结果
    all_results = {}

    # 对每种类型的文本进行评价
    for text_type, original_text in texts.items():
        print(f"\n评价{text_type}的翻译效果:")
        
        # 获取该类型的翻译结果
        translated_text = translated_texts.get(text_type, original_text)

        # 创建翻译结果字典
        translator_result = {best_translator: translated_text}

        # 获取该类型文本的专业术语词典
        current_domain_terms = domain_terms.get(text_type, None)

        # 评价翻译质量
        evaluation_results = evaluator.evaluate_translators(
            original_text, translator_result, current_domain_terms
        )

        # 保存评价结果
        all_results[text_type] = evaluation_results[best_translator]

        # 输出评价结果
        print(f"\n{best_translator}翻译评价结果:")
        print(f"  语义保留度: {evaluation_results[best_translator]['SP']:.4f}")
        print(f"  语法正确性: {evaluation_results[best_translator]['GC']:.4f}")
        print(f"  表达流畅性: {evaluation_results[best_translator]['EF']:.4f}")
        print(f"  术语准确性: {evaluation_results[best_translator]['TA']:.4f}")
        print(f"  总分: {evaluation_results[best_translator]['Total']:.4f}")

    return all_results
\end{lstlisting}

\subsection{附录三:评价模型中使用的翻译文本数据}
以下是评价模型中使用的各类翻译文本数据样例:

\subsubsection{问题一:双向翻译评价原始文本}
\begin{mdframed}
本校是一所以工为主、工理文协调发展的多科性全国重点大学,是国家"211工程"和"985工程"重点建设高校。
\end{mdframed}

\subsubsection{问题一:Google双向翻译结果}
\begin{mdframed}
我校是一所以工程为主、工程、科学和人文学科协调发展的多学科国家重点大学,是国家"211工程"和"985工程"重点建设高校。
\end{mdframed}

\subsubsection{问题一:百度双向翻译结果}
\begin{mdframed}
我们学校是一所以工科为主,以工科、理科和文科为协调发展的多学科全国重点大学。它是国家"211工程"和"985工程"的重点高校。
\end{mdframed}

\subsubsection{问题一:DeepL双向翻译结果}
\begin{mdframed}
本校是一所以工科为主、工科、理科和文科协调发展的多学科性质的国家重点大学,是国家"211工程"和"985工程"重点建设的高校。
\end{mdframed}

\subsubsection{问题二:英文短文原文}
\begin{mdframed}
Translation plays a crucial role in our globalized world. As communication barriers break down, the need for accurate and efficient translation has never been greater. This article discusses the evolution of translation technology from manual methods to advanced AI-powered systems.
\end{mdframed}

\subsubsection{问题三:唐诗原文及译文}
\begin{mdframed}
原文:\\
春眠不觉晓,\\
处处闻啼鸟。\\
夜来风雨声,\\
花落知多少。

译文:\\
一觉春宵不知晓,\\
处处闻鸟鸣。\\
夜风雨声,\\
不知落花几多。
\end{mdframed}

\subsubsection{问题三:数学教材原文及译文}
\begin{mdframed}
原文:\\
向量空间是数学中的一个基本概念。在线性代数中,向量空间是由向量组成的集合,这些向量可以进行加法运算和标量乘法运算,并满足八条向量公理。

译文:\\
向量空间是数学中的一个基本概念。在线性代数中,向量空间是一组可以与标量进行加法和乘法运算的向量,并且满足八个向量公理。
\end{mdframed}
\end{appendices}
\end{document}


%%%%%双图模板%%%%%%
\begin{figure}
\centering
\subcaptionbox{炉温曲线示意图\label{fig:双图a}}
{\includegraphics[width=.4\textwidth]{炉温曲线示意图.png}}
\subcaptionbox{问题1炉温曲线\label{fig:双图b}}
{\includegraphics[width=.4\textwidth]{问题1炉温曲线.png}}
\caption{双图}\label{fig:双图}
\end{figure}
%%%%%双图模板%%%%%%
