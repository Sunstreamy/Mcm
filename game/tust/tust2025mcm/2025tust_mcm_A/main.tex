%! Mode:: "TeX:UTF-8"
%! TEX program = xelatex
\PassOptionsToPackage{quiet}{xeCJK}
\documentclass[bwprint]{cumcmthesis}
\usepackage{etoolbox}
\BeforeBeginEnvironment{tabular}{\zihao{-5}}
\usepackage[numbers,sort&compress]{natbib}  % 文献管理宏包
\usepackage[framemethod=TikZ]{mdframed}  % 框架宏包
\usepackage{url}  % 网页链接宏包
\usepackage{subcaption}  % 子图宏包
\usepackage{amsmath} % 提供 \boldsymbol 命令
\usepackage{bm}      % 可选,增强粗体符号效果
\usepackage{amsfonts}     % 额外的数学字体
\usepackage{algorithm}    % 用于 algorithm 浮动环境
\usepackage{algorithmicx} % 核心算法排版功能
\usepackage{algpseudocode} % algorithmicx 的一种风格,提供 \State, \Function, \If 等命令
\usepackage{float}        % 用于 [H] 选项,强制浮动体位置
\newcolumntype{C}{>{\centering\arraybackslash}X}
\newcolumntype{R}{>{\raggedleft\arraybackslash}X}
\newcolumntype{L}{>{\raggedright\arraybackslash}X}


% 自定义信息
\newcommand{\studentid}{24103219} % 学号
\newcommand{\studentname}{贺江阳} % 姓名
\newcommand{\college}{人工智能学院} % 所在学院
\newcommand{\major}{软件工程} % 专业

% 重定义章节标题格式
\renewcommand{\thesection}{\chinese{section}、}
\renewcommand{\thesubsection}{\arabic{section}.\arabic{subsection}}
\renewcommand{\thesubsubsection}{\arabic{section}.\arabic{subsection}.\arabic{subsubsection}}

% 自定义首页格式
\renewcommand{\maketitle}{
  \begin{titlepage}
    \begin{center}
      \vspace*{2cm}
      {\huge\bfseries 基于多尺度特征与优化刚性配准的几何体鲁棒判别模型\par}
      \vspace{2cm}
      
      \begin{tabular}{rl}
        \Large\textbf{学 \quad 号:} & \Large\underline{\makebox[8cm]{\studentid}} \\[12pt]
        \Large\textbf{姓 \quad 名:} & \Large\underline{\makebox[8cm]{\studentname}} \\[12pt]
        \Large\textbf{所在学院:} & \Large\underline{\makebox[8cm]{\college}} \\[12pt]
        \Large\textbf{专 \quad 业:} & \Large\underline{\makebox[8cm]{\major}} \\[12pt]
      \end{tabular}
      
    \end{center}
  \end{titlepage}
}

% 重新定义目录样式
\renewcommand{\contentsname}{目录}

%%%%%%%%%%%%%%%%%%%%%%%%%%%%%%%%%%%%%%%%%%%%%%%%%%%%%%%%%%%%%
%% 正文
\begin{document}

% 显示封面
\maketitle

% 添加目录页
\cleardoublepage % 或者 \newpage,确保从新的一页开始
                 % \cleardoublepage 更好,因为它会处理好奇偶页

\pagestyle{empty}    % <--- 从这里开始,所有后续页面的样式都为空(无页眉页脚)
\tableofcontents     % 生成目录 (可能会跨越多页,所有这些页都将是 empty 样式)

\cleardoublepage     % 结束目录页,并确保下一页是奇数页,同时也让 empty 样式到此为止
\pagestyle{plain}    % <--- 恢复页面样式为 plain (通常页码在底部居中)
                     % 如果你的模板有特定的正文页面样式,例如 'headings',这里用那个
% 引入各个章节文件
% 摘要页
\begin{abstract}

开始摘要的主要内容

\keywords{几何体归类 \quad 刚性配准 \quad 多尺度特征提取 \quad Kabsch算法 \quad 迭代最近点(ICP)算法}

\end{abstract}
\include{mythesis/5_q1}
\include{mythesis/6_q2}
\include{mythesis/7_repair}

%%%%%%%%%%%%%%%%%%%%%%%%%%%%%%%%%%%%%%%%%%%%%%%%%%%%%%%%%%%%%
%% 参考文献

\nocite{*} % 引用所有.bib文件中的条目,即使未在正文中\cite
\bibliographystyle{gbt7714-numerical}  % 引用格式,符合GB/T 7714-2015数值引用
\bibliography{ref}  % bib源文件名 (ref.bib)
% 请确保您有一个名为 ref.bib 的文件,并在其中按上述格式或标准BibTeX格式管理您的参考文献。

\newpage
%%%%%%%%%%%%%%%%%%%%%%%%%%%%%%%%%%%%%%%%%%%%%%%%%%%%%%%%%%%%%
\phantomsection    % 为 hyperref 创建正确的锚点,使得目录中的“附录”可点击
\addcontentsline{toc}{section}{附录} % 将“附录”作为 section 级别添加到目录
                                                  % 你可以调整字体大小和粗细
\include{mythesis/9_appendix}
\newpage


\end{document}