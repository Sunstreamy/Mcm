% 摘要页
\section*{摘要}
\addcontentsline{toc}{section}{摘要} % 确保在目录中正确显示

针对观测几何体在未知姿态、尺度及噪声干扰下的精确自动归类挑战,本文提出一种多策略融合判别模型。其构建缘由在于:首先,通过重心平移及基于体积(或RMS距离)的尺度归一化进行预处理,为后续形状分析提供一致性基准;其次,为全面表征几何形态并提升辨识力,针对三维八面体构建了包含边长统计量、顶点至质心距离分布、体积与球形度、PCA惯量特征及角度统计量的21维多尺度特征集,而二维五边形则采用边长序列、内角序列、顶点到质心距离序列和面积构成的8维特征向量;再者,利用精确的刚性配准技术量化几何吻合度(RMSD),具体地,二维五边形采用结合循环旋转及镜像对称枚举的全局优化Kabsch算法,三维八面体则采用迭代最近点(ICP)算法,通过迭代优化实现鲁棒空间对齐;最后,为整合抽象特征相似性与几何配准精度,并客观确定分类边界,引入加权综合评分机制,并通过系统的参数优化流程来确定最优权重与阈值,从而有效解决题目中二维五边形和三维八面体的分类问题。

模型的核心计算流程为:对输入顶点进行归一化预处理,提取问题一(五边形)的8维特征向量和问题二(八面体)的21维特征向量,通过Kabsch全局优化算法(五边形)或ICP迭代算法(八面体)计算RMSD。最终分类依据为结合了特征向量欧氏距离$d_{\mathcal{F}}$与RMSD的加权综合评分 $S = \alpha \cdot \text{RMSD} + (1-\alpha) \cdot d_{\mathcal{F}}$,以及经参数优化确定的分类阈值$\lambda$。关键超参数$\alpha$和$\lambda$通过网格搜索并以最大化自定义综合评估函数为目标进行选取。

为评估模型性能,通过对标准几何体施加不同强度的高斯随机噪声和特定几何形变进行了系统的鲁棒性测试。结果显示模型具有良好的抗干扰能力:在平面五边形分类中,当噪声标准差低于0.20(相对单位)或形变程度低于平均边长的15\%时,对已知类别的平均分类准确率能维持在90\%以上;八面体分类模型在噪声标准差低于0.13或随机顶点位移小于10\%时,表现出90\%以上的平均准确率,特别是对单轴拉伸形变则可耐受20\%的形变程度。模型在低噪声区域($\sigma \leq 0.10$)能够为类别I和类别II样本维持95\%以上的识别率,同时对负样本始终保持90\%以上的正确拒绝率。两种模型计算效率均较高,单个样本处理时间均小于12毫秒。模型凭借其五边形Kabsch全局优化算法与多尺度特征组合、八面体21维特征向量与ICP迭代配准的有效结合,以及经过精心调优的综合评分与阈值判据,展现了卓越的空间几何体分类性能。

问题一回答:采用优化后的参数($\alpha_p=0.84, \lambda_p=0.79$),对表二中五个待测五边形的判别结果为:图形1、3、4被归类为类别I,图形2、5被归类为类别II。

问题二回答:采用优化后的参数($\alpha_o=0.81, \lambda_o=0.53$),对表四中六个观测八面体的判别结果为:八面体1被归类为类别I,八面体2、6被归类为类别II,而八面体3、4、5由于其最小综合评分仍大于优化确定的阈值,被判定为不属于任何已知类别。


\vspace{\fill} % 将关键词推到页面底部
\noindent % 确保关键词不缩进
\textbf{关键词:}几何体归类, 刚性配准, 多尺度特征提取, Kabsch算法, 迭代最近点(ICP)算法
\clearpage % 摘要结束后换页

\section{问题的分析}
对本文提出的几何体自动归类判别问题,我们针对其在二维平面和三维空间中的不同表现形式,逐一做如下分析:

问题一的分析(平面五边形):
解决平面五边形的归类问题,核心在于如何在输入顶点坐标可能存在平移、旋转及尺度不一致(如单位制差异)的情况下,准确识别其本质形状并与标准类别进行匹配。为此,我们首先需要对顶点数据进行标准化预处理,包括将几何中心(质心)平移至坐标原点,并采用如最大边长或周长等方式进行尺度归一化,以消除这些外部变换的干扰。在此基础上,我们将提取对旋转变换不敏感的多尺度几何不变量特征,例如排序后的边长序列、内角序列、顶点到质心的距离序列以及图形面积。这些特征共同构成了五边形的形状描述符。匹配阶段,可以采用Kabsch算法进行刚性配准,计算待测五边形与标准类别在最优对齐下的均方根误差(RMSD)。最终,通过结合特征向量间的相似度与RMSD设计一个综合评分机制,并设定合理的判别阈值,以实现精确归类。

问题二的分析(三维八面体):
三维八面体的归类问题在复杂性上相较于平面问题有所提升,主要体现在需要处理更高维度的空间变换以及更丰富的几何信息。其解决思路可视为问题一框架在三维空间的扩展与深化。同样需要进行重心和尺度的标准化预处理。在特征提取方面,除了可以借鉴二维的边长、顶点到质心距离等概念(扩展至三维的12条边、6个顶点),还应考虑三维特有的全局特征,如八面体的体积(可通过凸包计算)、主成分分析(PCA)得到的形状主轴方向及延展度、以及可能的对称性度量指标。对于三维点集的刚性配准,当顶点对应关系未知时,迭代最近点(ICP)算法是更为通用和强大的选择,它能够迭代地优化旋转和平移变换,最小化源点集与目标点集之间的距离。计算配准后的RMSD依然是衡量空间吻合度的有效手段。与问题一类似,最终的分类决策也将依赖于一个综合了特征相似度和空间配准误差的评分,并结合针对三维特性设定的判别阈值。问题二与问题一的核心思想(预处理-特征提取-配准-决策)是一致的,但具体采用的算法和特征集会因维度和几何体复杂度而有所调整。

\section{模型假设}
根据本文提出的问题和以上的问题分析,为确保后续模型构建的科学性和结果的有效性,我们做了如下关键模型假设:
\begin{enumerate}
    \item 几何不变量的稳定性:假设属于同一标准类别的几何体,在经历平移、旋转和(均匀)尺度等刚性变换后,其核心的、用于描述形状本质的几何不变量特征(如归一化后的相对边长比例、内角大小等)基本保持不变。
    \item 观测噪声的有限性:假设实际观测得到的顶点坐标数据中可能包含一定程度的随机噪声,但这些噪声的量级不足以从根本上改变几何体的基本拓扑结构(如顶点连接关系)和可识别的宏观形态。
    \item 分类阈值的有效性:假设存在一个或一组可以通过实验分析合理设定的判别阈值。当一个待测几何体与某个标准类别计算得到的综合相似度评分优于此阈值时,即可判定其属于该类别;反之,若与所有已知类别的评分均未达标,则判定为不属于任何已知类别。
    \item 顶点顺序的可处理性:对于依赖顶点顺序的算法(如二维Kabsch配准),假设可以通过预处理(如规范化排序)或特征提取方式(如对特征序列排序)来保证比较的一致性。对于三维ICP算法,其本身对初始顶点顺序不敏感。
    \item 几何体的基本形态一致性:假设所有待归类的几何体(无论是观测样本还是标准类别)在基本的拓扑形态上是明确的(例如,五边形均有5个顶点,八面体均有6个顶点和预期的面连接方式),模型主要处理的是这些顶点具体坐标位置上的差异。
    \item 标准类别的代表性与可区分性:假设所提供的标准类别几何体能够良好地代表其所属类别的典型形状,并且不同标准类别之间在模型所采用的特征和配准方法下具有足够的可区分度。
\end{enumerate}

\section{问题的提出}
\subsection{问题的重述}
对于平面或空间几何形状进行分析归类在天文学、图形图像学和材料科学等领域具有重要意义。由于观测手段、观测工具和采取单位制的不同,所观测的数据会有很大差别。利用计算机对这些数据进行处理并归类是当前的热点问题。本文将要解决以下几个问题:

\textbf{问题一} 已知平面上两类几何体均为五边形,在某直角坐标下,其各个顶点坐标(见表一),表二为观测到的五个五边形的顶点坐标,对表二的几何形状进行归类(该图形属于哪一类,或者不属于任何一类)。

\textbf{问题二} 已知三维空间上的两类八面体,在某直角坐标下,其各个顶点坐标(见表三),表四为观测到的六个八面体的各个顶点坐标,对表四的八面体进行归类(该图形属于哪一类,或者不属于任何一类)。

\subsection{研究背景}
\subsubsection*{(1) 研究意义}
几何形状分析与归类在计算机视觉、天文学、晶体学和医学成像等领域具有广泛应用。在天文学中,识别和分类天体的形态可以揭示宇宙的演化规律;在计算机视觉中,物体识别和场景理解依赖于对几何形状的准确归类;在材料科学中,晶体结构的分类有助于预测材料性质。随着观测技术的进步,大量不同尺度和形态的几何数据需要被有效分析。建立一个可靠的几何体归类判别模型,能够在尺度变化、旋转、平移和噪声扰动等情况下准确识别几何体的类别,对科学研究和工程应用具有重要意义。

\subsubsection*{(2) 参考文献综述}
几何体归类与点集配准问题已有多年研究历史。Kabsch\cite{kabsch1976solution}在1976年提出了一种通过最优旋转关联两组点集的算法,成为点集刚性配准的基础方法。Besl和McKay\cite{besl1992method}在1992年提出的迭代最近点(ICP)算法进一步解决了点集间的配准问题,广泛应用于三维形状匹配。Mardia和Kent\cite{mardia1987procrustes}发展了Procrustes分析方法用于形状的统计分析。近年来,Chen等人\cite{chen2015robust}和Ma等人\cite{ma2014robust}提出了一系列基于向量场一致性和稀疏空间共识的鲁棒特征匹配方法,提高了点集匹配在噪声和异常值存在情况下的精度。Zhang等人\cite{zhang2020recent}对点云配准的几何方法和基于学习的方法进行了全面综述。这些研究为本文构建多尺度特征与刚性配准结合的几何体归类判别模型提供了重要理论基础和技术支持。

\section{符号说明}
本文常用符号见下表,其它符号见文中说明。

\begingroup % 开始局部作用域
\setlength{\heavyrulewidth}{2.5pt} % 设置希望的粗细,仅在此组内有效

\begin{table}[H]
\centering
\caption{符号说明表}
\begin{tabularx}{\textwidth}{CCC}
\toprule % 这条线会使用上面设置的 1.2pt
\textbf{符号}    & \textbf{说明}    & \textbf{单位} \\
\midrule % 中间线的粗细 (由 \lightrulewidth 控制) 保持不变
\textbf{$P$, $Q$} & \textbf{点集} & \textbf{-} \\
$p_i$, $q_i$ & 点集中的点 & - \\
$R$ & 旋转矩阵 & - \\
$t$ & 平移向量 & - \\
$RMSD$ & 均方根偏差 & - \\
$\mathbf{c}$ & 重心 & - \\
$d_{ij}$ & 点$i$与点$j$之间的距离 & - \\
$\theta_i$ & 内角 & 弧度 \\
$\mathcal{F}$ & 特征向量 & - \\
$\lambda$ & 阈值参数 & - \\
$\delta$ & 形变参数 & - \\
$\sigma$ & 噪声强度 & - \\
\bottomrule % 这条线也会使用上面设置的 1.2pt
\end{tabularx}
\label{tab:符号说明}
\end{table}

\endgroup % 结束局部作用域,\heavyrulewidth 恢复到之前的值