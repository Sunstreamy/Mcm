\subsection{问题二的建模与求解}
% 简短的概述, 问题二主要要解决什么问题, 建立了××××××模型, 用了什么方法解决. 
% 写出本小节主要内容(比如:本小节主要内容是数据的预处理、××××××模型的建立、模型的求解和分析、模型检验).
问题二主要解决 ×××××× 问题,为此建立了 ×××××× 模型, 采用了 ×××××× 方法进行求解。
本小节主要内容包括:数据的预处理、××××××模型的建立、模型的求解和分析、模型检验或修正。

\subsubsection{数据的预处理(如果是做数据多的题, 一定要写这一块; 不需要的就不做.)}
    \textbf{1. 数据的采集}
    
    ××××××××
    
    \textbf{2. 去噪处理}
    
    ××××××××
    
    \textbf{3. 数据的整理(如不全的数据要补全, 有的是要转化数据类型, 等等)}
    
    ××××××××

\subsubsection{××××××模型的建立 (在此处为模型命名)}
    % 这一目给大家介绍编辑文字的注意事项:
    % 1. 中文用宋体, 正体, 小四号, 见图2. 数字和英文用Times new Roman. 
    %    变量, 函数, 公式一律用公式编辑器(Mathtype)编写(公式编辑器的默认小四号中文对应英文变量为Times new Roman, 斜体, 12.5号). 
    %    实际上, 数字最好也用公式编辑器, 整体效果会比较好. 
    %    如果在公式中要强调某个变量是向量或矩阵, 则要用Times new Roman, 粗体+斜体, 12.5号. 
    %    (a.txt中的图2是Word字体设置截图,此处为LaTeX注释)
    % 2. 行距用单倍行距. 或控制在固定值16, 和单倍行距效果相似. 
    %    (a.txt中的图3是Word行距设置截图,此处为LaTeX注释)
    % 3. 段落间有间距比较好看. 建议段前0.5倍行距. 
    % 4. 公式编辑器的调用:插入→对象→MathType→确定
    %    (a.txt中的图4是Word公式编辑器调用截图,此处为LaTeX注释)
    %    也可以直接将word文档里现有的公式编辑器编写的公式, 如 $E=mc^2$, 复制, 粘贴到需要的位置, 再双击打开, 编写. 
    % 注意公式一般有两种:隐式和显式. 隐式一般比较短小, 写在文字中间即可.
    % 显式是比较大或者要重点强调的公式, 要单独一行居中. 
    % 后面需要用到的公式还要编号. 如:
    % 相邻节点 $v_i$ 和 $v_{i+1}$ 之间满足关系式:
    % \begin{equation}
    %     y_{i+1} = A y_i + B u_i \label{eq:state_space_q2}
    % \end{equation}
    % 其中, $A, B$ 是系数矩阵, $u_i$ 是控制输入.
    在此处描述问题二的模型建立过程。
    例如:
    \begin{equation}
        \frac{d\mathbf{x}}{dt} = f(\mathbf{x}(t), \mathbf{u}(t), t) \label{eq:q2_model}
    \end{equation}
    其中 $\mathbf{x}$ 是状态向量, $\mathbf{u}$ 是控制向量。

\subsubsection{模型的求解和分析}
    \textbf{1.参数的确定与模型求解}
    
    通过xx方法,确定公式(\ref{eq:q2_model})中的参数. 运用×××软件的什么命令, 或自己编写的程序(程序见附录×), 得到怎样的结果. 结果展示除数据公式外也可利用图或表. 
    
    \textbf{2. 结果分析}
    
    从图×或表×可以看出:×××××××××××××××. 结论是什么.
    从图×或表×可以看出:×××××××××××××××(单调性,极值,最值,拐点). 结论是什么(定性结果:变好、变坏,增加、减少).
    
    \textbf{3. 问题二的回答}
    
    利用计算结果,逐一回答问题二提出的每一个问题。对随时间变换的量,建议给出时间变换曲线图。

\subsubsection{模型检验或修正}
说明假设合理性,说明在前文的假设下结果的正确性(合理性), 诸如此类.可以通过对误差进行分析(如,做仿真模拟), 提出修改假设和对模型的修正。
××××××
