\section{问题的提出}
\subsection{问题的重述}
在此处重述赛题问题。例如:
××××××问题就是要解决××××××××××××事情。
本文将要解决以下几个问题:

\textbf{问题一} (将论文题目的问题一摘抄过来即可)

\textbf{问题二} (将论文题目的问题二摘抄过来即可)

\textbf{问题三} (将论文题目的问题三摘抄过来即可)

\subsection{研究背景}
\subsubsection*{(1) 研究意义} % 使用无编号的subsubsection以匹配a.txt格式
% 结合题目和资料, 交代一下研究问题的重要性\研究意义.(字数不要太多,200-400左右.)
在此处填写研究意义。××××××

\subsubsection*{(2) 参考文献综述}
几何体归类与点集配准问题已有多年研究历史。Kabsch\cite{kabsch1976solution}在1976年提出了一种通过最优旋转关联两组点集的算法,成为点集刚性配准的基础方法。Besl和McKay\cite{besl1992method}在1992年提出的迭代最近点(ICP)算法进一步解决了点集间的配准问题,广泛应用于三维形状匹配。Mardia和Kent\cite{mardia1987procrustes}发展了Procrustes分析方法用于形状的统计分析。近年来,Chen等人\cite{chen2015robust}和Ma等人\cite{ma2014robust}提出了一系列基于向量场一致性和稀疏空间共识的鲁棒特征匹配方法,提高了点集匹配在噪声和异常值存在情况下的精度。Zhang等人\cite{zhang2020recent}对点云配准的几何方法和基于学习的方法进行了全面综述。这些研究为本文构建多尺度特征与刚性配准结合的几何体归类判别模型提供了重要理论基础和技术支持。



