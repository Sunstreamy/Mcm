\begin{appendices} % 开始附录环境
    % 这个环境会在目录中添加"附录"这一行,并在正文打印"附录"大标题。
    

\begin{table}[H]
\centering
% 列定义:最左、最右有竖线,并且在第一列和第二列之间,第二列和第三列之间也有竖线
\begin{tabularx}{\linewidth}{| >{\centering}p{0.15\linewidth} | L | L |} % <--- 修改列定义
\toprule
% 跨三列的“附录清单”标题,左右也应该有竖线以匹配表格外框
\multicolumn{3}{|c|}{\textbf{\Large 附录清单}} \\ % 注意这里multicolumn的格式调整为 c| 或 |c| 或 |c
                                                % 更精确的应该是 |c| 如果左右都有线
                                                % 但由于它已经是最外层,tabularx的 | 会处理外框
                                                % 不过,为了清晰,且如果下面有内部竖线,
                                                % 通常 \multicolumn 也需要指定其内部竖线。
                                                % 这里因为附录清单下面没有内部竖线,所以 c 就可以。
                                                % 但为了应对下一行的竖线,我们让它也匹配竖线模式
\midrule
% 列标题,列与列之间有竖线
\textbf{附录编号} & \textbf{名称} & \textbf{解释} \\ % 这里的 & 会自动在它们之间画线 (基于上面的 |L|L|)
\midrule
附录A & 可视化函数(部分) & 可视化数据 \\
\midrule % 在每个数据行后加一条水平线,以模仿你图片中每行都有分隔的效果
附录B & 问题一:五边形分类代码(主要逻辑) & q1.py 文件中与五边形分类相关的核心类和函数定义 \\
\midrule
附录C & 问题二:八面体分类代码(主要逻辑) & q2.py 文件中与八面体分类相关的核心类和函数定义 \\
\bottomrule
\end{tabularx}
\end{table}
    
\subsection*{附录A: 可视化函数}
    
\noindent visual.py
\lstinputlisting[language=python,basicstyle=\ttfamily\small]{code/visual.py}
    
\subsection*{附录B: 问题一五边形分类核心代码}
    
\noindent q1.py
\lstinputlisting[language=python, basicstyle=\ttfamily\small]{code/q1.py}

\subsection*{附录C: 问题二八面体分类核心代码}
    
\noindent q2.py
\lstinputlisting[language=python,  basicstyle=\ttfamily\small]{code/q2.py}
    
\end{appendices} % 结束附录环境