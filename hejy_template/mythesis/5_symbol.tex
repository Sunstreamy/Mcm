\section{符号说明}
本文常用符号见下表,其它符号见文中说明。

\begingroup % 开始局部作用域
\setlength{\heavyrulewidth}{2.5pt} % 设置希望的粗细,仅在此组内有效

\begin{table}[H]
\centering
\caption{符号说明表}
\begin{tabularx}{\textwidth}{CCC}
\toprule % 这条线会使用上面设置的 1.2pt
\textbf{符号}    & \textbf{说明}    & \textbf{单位} \\
\midrule % 中间线的粗细 (由 \lightrulewidth 控制) 保持不变
\textbf{$P$, $Q$} & \textbf{点集} & \textbf{-} \\
$p_i$, $q_i$ & 点集中的点 & - \\
$R$ & 旋转矩阵 & - \\
$t$ & 平移向量 & - \\
$RMSD$ & 均方根偏差 & - \\
$\mathbf{c}$ & 重心 & - \\
$d_{ij}$ & 点$i$与点$j$之间的距离 & - \\
$\theta_i$ & 内角 & 弧度 \\
$\mathcal{F}$ & 特征向量 & - \\
$\lambda$ & 阈值参数 & - \\
$\delta$ & 形变参数 & - \\
$\sigma$ & 噪声强度 & - \\
\bottomrule % 这条线也会使用上面设置的 1.2pt
\end{tabularx}
\label{tab:符号说明}
\end{table}

\endgroup % 结束局部作用域,\heavyrulewidth 恢复到之前的值